\documentclass[11pt]{article}

\usepackage{color}
%\input{rgb}

%----------Packages----------
\usepackage{amsmath}
\usepackage{amssymb}
\usepackage{amsthm}
\usepackage{amsrefs}
\usepackage{dsfont}
\usepackage{mathrsfs}
\usepackage{stmaryrd}
\usepackage[all]{xy}
\usepackage[mathcal]{eucal}
\usepackage{verbatim}  %%includes comment environment
\usepackage{fullpage}  %%smaller margins
%----------Commands----------

%%penalizes orphans
\clubpenalty=9999
\widowpenalty=9999





%% bold math capitals
\newcommand{\bA}{\mathbf{A}}
\newcommand{\bB}{\mathbf{B}}
\newcommand{\bC}{\mathbf{C}}
\newcommand{\bD}{\mathbf{D}}
\newcommand{\bE}{\mathbf{E}}
\newcommand{\bF}{\mathbf{F}}
\newcommand{\bG}{\mathbf{G}}
\newcommand{\bH}{\mathbf{H}}
\newcommand{\bI}{\mathbf{I}}
\newcommand{\bJ}{\mathbf{J}}
\newcommand{\bK}{\mathbf{K}}
\newcommand{\bL}{\mathbf{L}}
\newcommand{\bM}{\mathbf{M}}
\newcommand{\bN}{\mathbf{N}}
\newcommand{\bO}{\mathbf{O}}
\newcommand{\bP}{\mathbf{P}}
\newcommand{\bQ}{\mathbf{Q}}
\newcommand{\bR}{\mathbf{R}}
\newcommand{\bS}{\mathbf{S}}
\newcommand{\bT}{\mathbf{T}}
\newcommand{\bU}{\mathbf{U}}
\newcommand{\bV}{\mathbf{V}}
\newcommand{\bW}{\mathbf{W}}
\newcommand{\bX}{\mathbf{X}}
\newcommand{\bY}{\mathbf{Y}}
\newcommand{\bZ}{\mathbf{Z}}

%% blackboard bold math capitals
\newcommand{\bbA}{\mathbb{A}}
\newcommand{\bbB}{\mathbb{B}}
\newcommand{\bbC}{\mathbb{C}}
\newcommand{\bbD}{\mathbb{D}}
\newcommand{\bbE}{\mathbb{E}}
\newcommand{\bbF}{\mathbb{F}}
\newcommand{\bbG}{\mathbb{G}}
\newcommand{\bbH}{\mathbb{H}}
\newcommand{\bbI}{\mathbb{I}}
\newcommand{\bbJ}{\mathbb{J}}
\newcommand{\bbK}{\mathbb{K}}
\newcommand{\bbL}{\mathbb{L}}
\newcommand{\bbM}{\mathbb{M}}
\newcommand{\bbN}{\mathbb{N}}
\newcommand{\bbO}{\mathbb{O}}
\newcommand{\bbP}{\mathbb{P}}
\newcommand{\bbQ}{\mathbb{Q}}
\newcommand{\bbR}{\mathbb{R}}
\newcommand{\bbS}{\mathbb{S}}
\newcommand{\bbT}{\mathbb{T}}
\newcommand{\bbU}{\mathbb{U}}
\newcommand{\bbV}{\mathbb{V}}
\newcommand{\bbW}{\mathbb{W}}
\newcommand{\bbX}{\mathbb{X}}
\newcommand{\bbY}{\mathbb{Y}}
\newcommand{\bbZ}{\mathbb{Z}}

%% script math capitals
\newcommand{\sA}{\mathscr{A}}
\newcommand{\sB}{\mathscr{B}}
\newcommand{\sC}{\mathscr{C}}
\newcommand{\sD}{\mathscr{D}}
\newcommand{\sE}{\mathscr{E}}
\newcommand{\sF}{\mathscr{F}}
\newcommand{\sG}{\mathscr{G}}
\newcommand{\sH}{\mathscr{H}}
\newcommand{\sI}{\mathscr{I}}
\newcommand{\sJ}{\mathscr{J}}
\newcommand{\sK}{\mathscr{K}}
\newcommand{\sL}{\mathscr{L}}
\newcommand{\sM}{\mathscr{M}}
\newcommand{\sN}{\mathscr{N}}
\newcommand{\sO}{\mathscr{O}}
\newcommand{\sP}{\mathscr{P}}
\newcommand{\sQ}{\mathscr{Q}}
\newcommand{\sR}{\mathscr{R}}
\newcommand{\sS}{\mathscr{S}}
\newcommand{\sT}{\mathscr{T}}
\newcommand{\sU}{\mathscr{U}}
\newcommand{\sV}{\mathscr{V}}
\newcommand{\sW}{\mathscr{W}}
\newcommand{\sX}{\mathscr{X}}
\newcommand{\sY}{\mathscr{Y}}
\newcommand{\sZ}{\mathscr{Z}}

\newcommand{\fr}[2]{\frac{\underline{#1}}{#2}}


\renewcommand{\phi}{\varphi}

\renewcommand{\emptyset}{\O}

\providecommand{\abs}[1]{\lvert #1 \rvert}
\providecommand{\norm}[1]{\lVert #1 \rVert}


\providecommand{\ar}{\rightarrow}
\providecommand{\arr}{\longrightarrow}

\renewcommand{\_}[1]{\underline{ #1 }}


\DeclareMathOperator{\ext}{ext}


\newcommand{\head}[1]{
	\begin{center}
		{\large #1}
		\vspace{.2 in}
	\end{center}
	
	\bigskip 
}

%----------Theorems----------

\newtheorem{theorem}{Theorem}[section]
\newtheorem{proposition}[theorem]{Proposition}
\newtheorem{lemma}[theorem]{Lemma}
\newtheorem{corollary}[theorem]{Corollary}


\newtheorem*{axiom4}{Axiom 4}


\theoremstyle{definition}
\newtheorem{definition}[theorem]{Definition}
\newtheorem{nondefinition}[theorem]{Non-Definition}
\newtheorem{exercise}[theorem]{Exercise}
\newtheorem{remark}[theorem]{Remark}
\newtheorem{warning}[theorem]{Warning}
\newtheorem{examples}[theorem]{Examples}
\newtheorem{example}[theorem]{Example}


\newcommand{\set}[1]{\left\lbrace #1 \right\rbrace}
\newcommand{\N}{\mathbb{N}}
\newcommand{\Z}{\mathbb{Z}}
\newcommand{\Q}{\mathbb Q}
\newcommand{\st}{\ |\ }
\newcommand{\Hskip}{\vspace{0.7in}}
\numberwithin{equation}{subsection}
\newcommand{\hide}[1]{{\color{red} #1}} % for instructor version
\newcommand{\com}[1]{{\color{blue} #1}} % for instructor version


\begin{document}

\head{MATH 161, Autumn 2024\\ SCRIPT 2: The Rationals} 





%%---  sheet number for theorem counter
\setcounter{section}{2}   


Note that all of the following proofs should be straightforward consequences of properties of the integers; you should be sure to make it clear what facts about the integers you are using. 
See Script 0 for a list of the defining properties of $\bbZ$. 

 
\begin{definition}
Let $X$ be a nonempty set. A {\em relation} $R$ on $X$ is a subset of $X\times X.$ The statement $(x,y)\in R$ is read as `$x$ is related to $y$ by the relation $R,$' and is often denoted $x\sim y.$ \medskip

\noindent A relation is {\em reflexive} if $x\sim x$ for all $x\in X.$\\
A relation is {\em symmetric} if $y\sim x$ whenever $x\sim y.$\\
A relation is {\em transitive} if  $x\sim z$ whenever $x\sim y$ and $y\sim z.$ \\
A relation is an {\em equivalence relation} if it is reflexive, symmetric and transitive.\medskip
\end{definition}

\begin{exercise}\label{equivclassex}
Determine which of the following are equivalence relations.
\begin{enumerate}
\item[a)] Any set $X$ with the relation $=.$ So $x\sim y$ if and only if $x=y.$

True, since $x=x$ so it is reflexive, $x =y$ implies $y=x$ so it is symmetric, and if $x=y$ and $y=z$ then $x=z$ so it is transitive. Hence it is an equivalence relation.
\item[b)] $\mathbb{Z}$ with the relation $<.$ 

False, since $1 \not < 1$ so it is not reflexive (it is symmetric and not transitive).
\item[c)] Any subset $X $ of $\mathbb{Z}$ with the relation $\leq.$ So $x\sim y$ if and only if $x\leq y.$ 

False, since $1 \leq 2$ but $2 \not \leq 1$ so it is not symmetric (it is reflexive and transitive; it is symmetric in the special cases where $X = \emptyset$ or $|X|=1$).
\item[d)] $X=\mathbb{Z}$ with $x\sim y$ if and only if $y-x$ is divisible by 5.

True, since $x-x=0$ is divisible by 5 so it is reflexive. $y-x = -(x-y)$ so if $y-x$ is divisible by 5, then $x-y$ is divisible by 5, so it is symmetric. If $\frac{y-x}{5} \mod 1 \equiv 0$ and $\frac{z-y}{5} \mod 1 \equiv 0$, then $\frac{z-x}{5} \mod 1 = (\frac{z-y}{5} + \frac{y-x}{5}) \mod 1 \equiv 0$ so it is transitive.

\item[e)]  $X=\{(a,b)\mid a,b\in \mathbb{Z}, b\neq 0\}$ with the relation $\sim$ defined by
$$(a,b)\sim (c,d)\hspace{10pt}\mbox{if and only if}\hspace{10pt}ad=bc.$$

True, since $ab=ba$ so it is reflexive, $ad=bc$ implies $cb=da$ so it is symmetric, and if $ad=bc$ and $cf=de$ then $af=be$ since $adcf=bcde$, so it is transitive. Hence it is an equivalence relation.


 \end{enumerate}
 \end{exercise}


\begin{remark}  
A {\em partition} of a set is a collection of non-empty
disjoint subsets whose union is the original set.  
Any equivalence relation on a set creates a partition of that set by collecting into subsets all
of the elements that are equivalent (related) to each other.
When the partition of a set arises from an equivalence relation in this manner, the subsets
are referred to as {\em equivalence classes}. (See Exercises 2 and 3 in the Additional Exercises section below.)
\end{remark}  

\begin{remark}  
If we think of the set $X$ in \ref{equivclassex}e) as representing the collection of all fractions whose denominators are not
zero, then the relation $\sim$ may be thought of as representing the equivalence of two fractions.
\end{remark}  

\begin{definition}
As a set, the {\em rational numbers}, denoted $\bbQ$, are the equivalence classes in the set $X=\{(a,b)\mid a,b\in \mathbb{Z}, b\neq 0\}$ under
the equivalence relation $\sim$ as defined in \ref{equivclassex}e).
If $(a,b)\in X$, we denote the equivalence class of this element as $\left[ \frac{a}{b}\right]$.   So
$$\left[\frac{a}{b}\right]=\{(x_1,x_2)\in X\mid (x_1,x_2)\sim (a,b)\}=\{(x_1,x_2)\in X\mid x_1 b=x_2 a\}.$$
Then,
$$
\bbQ=\left\{ {\left[\frac{a}{b}\right]} \mid (a,b)\in X\right\}.
$$
\end{definition}

\begin{exercise}
$\displaystyle \left[\frac{a}{b}\right]=\displaystyle \left[\frac{a'}{b'}\right]\Longleftrightarrow (a,b)\sim (a',b').$
\begin{proof}
Let $x, x' \in [\frac{a}{b}]$. Then, $(x,x') \sim (a,b)$. By transitivity, we know that $(x,x')\sim(a',b')$. Therefore, $(x,x') \in [\frac{a'}{b'}]$. Hence, $[\frac{a}{b}] \subseteq [\frac{a'}{b'}]$. Analogously, $[\frac{a'}{b'}] \subseteq [\frac{a}{b}]$. Therefore, $[\frac{a'}{b'}] = [\frac{a}{b}]$, hence $(a,b) \sim (a',b')$.

\renewcommand\qedsymbol{QED}
\end{proof}

\end{exercise}

\begin{definition}
We define the binary operations addition and multiplication on $\bbQ$ as follows.  If $\displaystyle \left[ \frac{a}{b}\right]$ and $\displaystyle \left[\frac{c}{d}\right]$ are in $\bbQ$, then:
$$
\left[\frac{a}{b}\right]+_\bbQ \left[\frac{c}{d}\right]=\left[\frac{ad+bc}{bd}\right]
$$
$$
\left[\frac{a}{b}\right]\cdot_\bbQ \left[\frac{c}{d}\right]=\left[\frac{ac}{ bd}\right].
$$
We use the notation $+_\bbQ$ and $\cdot_\bbQ$ to represent addition and multiplication in $\bbQ$
so as to distinguish these operations from the usual addition ($+$) and multiplication ($\cdot$) in $\bbZ$.
\end{definition}


We want to know that addition and multiplication are ``well-defined'', by which we mean that if we change the representatives of the
classes $\left[\frac{a}{b}\right]$ and $\left[\frac{c}{d}\right]$, then does this not change the
resulting classes on the right-hand-side of the equalities in the definition.  To prove the next theorem first you will need to formulate a precise statement about what needs to be checked.

\begin{theorem}
Addition and multiplication in $\bbQ$ are well-defined.  
\begin{proof}[Proof for Addition]
We want to show that if $(a,b) \sim (a',b')$ and $(c,d) \sim (c',d')$ then $[\frac{ad+bc}{bd}] = [\frac{a'd'+b'c'}{b'd'}]$. Hence, we want to show that $(ad+bc)b'd' = (a'd'+b'c')bd$.

The LHS above is equivalent to $adb'd' + bcb'd' = ab'dd' + cd'bb' = a'bdd' + c'dbb' = (a'd'+b'c')bd$ = RHS. This completes the proof.


\renewcommand\qedsymbol{QED}
\end{proof}

\begin{proof}[Proof for Multiplication]
We want to show that if $(a,b) \sim (a',b')$ and $(c,d) \sim (c',d')$ then $[\frac{ac}{bd}] = [\frac{a'c'}{b'd'}]$. Hence, we want to show that $acb'd' = a'c'bd$.

The LHS above is equivalent to $acb'd' = a'bcd' = a'bc'd = a'c'bd$ = RHS. This completes the proof.


\renewcommand\qedsymbol{QED}
\end{proof}

\end{theorem}

\begin{theorem} 

 \begin{enumerate}
   \item[a)] {\bf Commutativity of addition}
   
   $\displaystyle \left[\frac{a}{b}\right]+_{\bbQ} \left[\frac{c}{d}\right] =\left[\frac{c}{d}\right]+_{\bbQ}\left[\frac{a}{b}\right]$ for all $\displaystyle  \left[\frac{a}{b}\right],\left[\frac{c}{d}\right] \in\bbQ.$

\begin{proof}
LHS = $[\frac{a}{b}] + [\frac{c}{d}] = [\frac{ad+bc}{bd}]$.

RHS = $[\frac{c}{d}] + [\frac{a}{b}] = [\frac{cb+da}{db}]$. Hence, LHS = RHS.

\renewcommand\qedsymbol{QED}
\end{proof}

  \item[b)] {\bf Associativity of addditon}
  
  $\displaystyle \left(\left[\frac{a}{b}\right]+_{\bbQ}\left[\frac{c}{d}\right] \right)+_{\bbQ}\left[\frac{e}{f}\right] = 
  \left[\frac{a}{b}\right]+_{\bbQ}\left( \left[\frac{c}{d}\right] +_{\bbQ} \left[\frac{e}{f}\right]\right) $ for all $\displaystyle \left[\frac{a}{b}\right],\left[\frac{c}{d}\right] , \left[\frac{e}{f}\right]\in \bbQ.$ 

\begin{proof}
LHS = $[\frac{ad+bc}{bd}] + [\frac{e}{f}] = [\frac{adf+bcf+bde}{bdf}] = [\frac{a}{b}] + [\frac{cf+de}{df}]$ = RHS.

\renewcommand\qedsymbol{QED}
\end{proof}

  \item[c)] {\bf Existence of an additive identity}
  
  $\displaystyle \left[\frac{a}{b}\right]+_{\bbQ}\left[\frac{0}{1}\right]=\left[\frac{a}{b}\right],$ for all $\displaystyle \left[\frac{a}{b}\right]\in\bbQ.$

\begin{proof}
LHS = $[\frac{a \cdot 1 + b \cdot 0}{b \cdot 1}] = [\frac{a}{b}]$ = RHS.

\renewcommand\qedsymbol{QED}
\end{proof}
  
  \item[d)] {\bf Existence of additive inverses}
  
  $\displaystyle \left[\frac{a}{b}\right]+_{\bbQ}\left[\frac{-a}{b}\right]=\left[\frac{0}{1}\right],$ for all $\displaystyle \left[\frac{a}{b}\right] \in \bbQ.$

\begin{proof}
LHS = $[\frac{a \cdot b + (-a) \cdot b}{b \cdot b}] = [\frac{ab-ab}{b^2}] = [\frac{0}{b^2}] = [\frac{0}{1}]$ = RHS.

\renewcommand\qedsymbol{QED}
\end{proof}

  \item[e)] {\bf Commutativity of multiplication}
  
  $\displaystyle \left[\frac{a}{b}\right]\cdot_{\bbQ} \left[\frac{c}{d}\right] =\left[\frac{c}{d}\right]\cdot_{\bbQ} \left[\frac{a}{b}\right]$ for all $\displaystyle  \left[\frac{a}{b}\right],\left[\frac{c}{d}\right] \in\bbQ.$

\begin{proof}
LHS = $[\frac{a \cdot c}{b \cdot d}] = [\frac{c \cdot a}{d \cdot b}]$ = RHS by commutativity of multiplication in $\Z$.

\renewcommand\qedsymbol{QED}
\end{proof}


  \item[f)] {\bf Associativity of multiplication}
  
   $\displaystyle \left(\left[\frac{a}{b}\right]\cdot_{\bbQ} \left[\frac{c}{d}\right] \right)\cdot_{\bbQ} \left[\frac{e}{f}\right] = 
  \left[\frac{a}{b}\right]\cdot_{\bbQ} \left( \left[\frac{c}{d}\right] \cdot_{\bbQ} \left[\frac{e}{f}\right]\right) $ for all $\displaystyle \left[\frac{a}{b}\right],\left[\frac{c}{d}\right] , \left[\frac{e}{f}\right]\in \bbQ.$ 

\begin{proof}
LHS = $[\frac{a \cdot c}{b \cdot d}] \cdot [\frac{e}{f}] = [\frac{a \cdot c \cdot e}{b \cdot d \cdot f}] = [\frac{c \cdot e}{d \cdot f}] \cdot [\frac{a}{b}]$ = RHS.

\renewcommand\qedsymbol{QED}
\end{proof}

    \item[g)]  {\bf Existence of a multiplicative identity}
    
     $\displaystyle \left[\frac{a}{b}\right] \cdot_{\bbQ}\left[\frac{1}{1}\right]=\left[\frac{a}{b}\right],$ for all $\displaystyle \left[\frac{a}{b}\right] \in\bbQ.$
     
\begin{proof}
LHS = $[\frac{a \cdot 1}{b \cdot 1}] = [\frac{a}{b}]$ = RHS.

\renewcommand\qedsymbol{QED}
\end{proof}

    \item[h)] {\bf Existence of multiplicative inverses for nonzero elements}
    
    $\displaystyle \left[\frac{a}{b}\right] \cdot_{\bbQ}\left[\frac{b}{a}\right]=\left[\frac{1}{1}\right],$ for all $\displaystyle \left[\frac{a}{b}\right] \in\bbQ$ such that $\displaystyle\left[\frac{a}{b}\right]\neq \left[\frac{0}{1}\right].$ 
    
\begin{proof}
LHS = $[\frac{a \cdot b}{b \cdot a}] = [\frac{1}{1}]$ = RHS, since $(ab)(1) = (ba)(1)$, hence $(ab,ba)\sim (1,1)$.

\renewcommand\qedsymbol{QED}
\end{proof}

  \item[i)] {\bf Distributivity} 
  
  $\displaystyle \left[\frac{a}{b}\right]\cdot_{\bbQ} \left(\left[\frac{c}{d}\right]+_{\bbQ}\left[\frac{e}{f}\right]\right)=\left(\left[\frac{a}{b}\right]\cdot_{\bbQ} \left[\frac{c}{d}\right]\right) +_{\bbQ} \left( \left[\frac{a}{b}\right]\cdot_{\bbQ} \left[\frac{e}{f}\right]\right),$ for all $\displaystyle \left[\frac{a}{b}\right],\left[\frac{c}{d}\right], \left[\frac{e}{f}\right] \in\bbQ.$  

\begin{proof}
LHS = $[\frac{a}{b}] \cdot [\frac{cf+de}{df}] = [\frac{acf+ade}{bdf}] = [\frac{acbf + aebd}{bdbf}] = [\frac{ac}{bd}] + [\frac{ae}{bf}]$ = RHS.

\renewcommand\qedsymbol{QED}
\end{proof}

   \end{enumerate}    



\end{theorem}


\begin{theorem} 

$\Q$ is countable.
\end{theorem}
{\em Hint: look back at Script 1} \bigskip
\begin{proof}
Let $f \colon \Q \rightarrow \bbZ \times \bbN$ be defined as $f([\frac{a}{b}]) = (a,b)$, for $[\frac{a}{b}] \in \Q$, where $a \in \Z$, $b \in \N$.

We know that $\bbZ \times \bbN$ is countable, hence it suffices to show that $f$ is injective.

Suppose $(a,b) \not \sim (a',b')$, i.e., $ab' \not= a'b$. 

Case 1: $a=a', b\not= b'$. Then, $f([\frac{a}{b}]) = (a,b) \not= (a,b') = f([\frac{a'}{b'}])$.

Case 2: $a\not=a', b= b'$. Then, $f([\frac{a}{b}]) = (a,b) \not= (a',b) = f([\frac{a'}{b'}])$.

Case 3: $a\not=a', b\not= b'$. Then, $f([\frac{a}{b}]) = (a,b) \not= (a',b') = f([\frac{a'}{b'}])$, since we know that $ab' \not= a'b$.

\renewcommand\qedsymbol{QED}
\end{proof}

We will now lose the equivalence class notation and simply refer to elements of $\Q$ as usual. So for example, if we refer to $0,$ we really mean $[
\frac{0}{1}],$ but this distinction should no longer be necessary or relevant.
\bigskip

\begin{center}
{\em Additional Exercises}
\end{center}

{\em In all exercises you are expected to prove your answer, unless explicitly stated otherwise.}


\begin{enumerate}

\item
Prove that for every $q=\left[\frac{a}{b}\right]\in\bbQ,$ there is exactly one element $(a_0,b_0)\in q$ such that $a_0$ and $b_0$ have no common factors and $b_0>0$.


\begin{proof}
First, we want to show that there is at least one such element. 

If $b<0$, then we know that $q=[\frac{a'}{b'}]$ for some $a' \in \Z, b' \in \N$.

Next, if $gcd(a',b')=1$, then we are done. If $gcd(a',b')=m>1$, then let $a_0 = \frac{a'}{m}$ and let $b_0 = \frac{b'}{m}$.

We know that $a_0 \cdot b' = \frac{a'b'}{m} = b_0 \cdot a'$, hence $(a_0,b_0) \sim (a',b')$. Therefore, we can write $q=[\frac{a_0}{b_0}]$.

Second, we want to show that there is at most one such element.

Suppose there are two such elements, $(a_0,b_0)$ and $(a_1,b_1)$ such that $a_0 \not = a_1, b_0 \not = b_1$. Since $gcd(a_0,b_0)=gcd(a_1,b_1)=1$, this implies $a_0 \cdot b_1 \not = a_1 \cdot b_0$. Then, by our initial assumption, $q = [\frac{a_0}{b_0}]$ and $q = [\frac{a_1}{b_1}]$, hence $(a_0,b_0) \sim (a_1,b_1)$, which implies $a_0 \cdot b_1 = a_1 \cdot b_0$ and this is a contradiction.

\renewcommand\qedsymbol{QED}
\end{proof}



\item Let $\sim$ be an equivalence relation on $X$. Let $[x]$ be the equivalence class of $x$. Show that
\begin{enumerate}
\item $[x] = [y]$ if and only if $x\sim y$
\item for all $x, y \in X$,
\[[x] \cap [y] = \begin{cases} [x] =[y] & \text{if $x \sim y$} \\
\emptyset & \text{if $x \not\sim y $}.
\end{cases}\]
\item $X = \displaystyle \bigcup_{x \in X}[x]$
\end{enumerate}
    
    


\item For the equivalence relations found in Exercise~\ref{equivclassex}, what are the equivalence classes?


\end{enumerate}


\end{document}

