\documentclass[11pt]{article}
\usepackage{color}
%\input{rgb}

%----------Packages----------
\usepackage{amsmath}
\usepackage{amssymb}
\usepackage{amsthm}
%\usepackage{amsrefs}
\usepackage{dsfont}
\usepackage{mathrsfs}
\usepackage{stmaryrd}
\usepackage[all]{xy}
\usepackage[mathcal]{eucal}
\usepackage{verbatim}  %%includes comment environment
\usepackage{fullpage}  %%smaller margins
%----------Commands----------

%%penalizes orphans
\clubpenalty=9999
\widowpenalty=9999





%% bold math capitals
\newcommand{\bA}{\mathbf{A}}
\newcommand{\bB}{\mathbf{B}}
\newcommand{\bC}{\mathbf{C}}
\newcommand{\bD}{\mathbf{D}}
\newcommand{\bE}{\mathbf{E}}
\newcommand{\bF}{\mathbf{F}}
\newcommand{\bG}{\mathbf{G}}
\newcommand{\bH}{\mathbf{H}}
\newcommand{\bI}{\mathbf{I}}
\newcommand{\bJ}{\mathbf{J}}
\newcommand{\bK}{\mathbf{K}}
\newcommand{\bL}{\mathbf{L}}
\newcommand{\bM}{\mathbf{M}}
\newcommand{\bN}{\mathbf{N}}
\newcommand{\bO}{\mathbf{O}}
\newcommand{\bP}{\mathbf{P}}
\newcommand{\bQ}{\mathbf{Q}}
\newcommand{\bR}{\mathbf{R}}
\newcommand{\bS}{\mathbf{S}}
\newcommand{\bT}{\mathbf{T}}
\newcommand{\bU}{\mathbf{U}}
\newcommand{\bV}{\mathbf{V}}
\newcommand{\bW}{\mathbf{W}}
\newcommand{\bX}{\mathbf{X}}
\newcommand{\bY}{\mathbf{Y}}
\newcommand{\bZ}{\mathbf{Z}}

%% blackboard bold math capitals
\newcommand{\bbA}{\mathbb{A}}
\newcommand{\bbB}{\mathbb{B}}
\newcommand{\bbC}{\mathbb{C}}
\newcommand{\bbD}{\mathbb{D}}
\newcommand{\bbE}{\mathbb{E}}
\newcommand{\bbF}{\mathbb{F}}
\newcommand{\bbG}{\mathbb{G}}
\newcommand{\bbH}{\mathbb{H}}
\newcommand{\bbI}{\mathbb{I}}
\newcommand{\bbJ}{\mathbb{J}}
\newcommand{\bbK}{\mathbb{K}}
\newcommand{\bbL}{\mathbb{L}}
\newcommand{\bbM}{\mathbb{M}}
\newcommand{\bbN}{\mathbb{N}}
\newcommand{\bbO}{\mathbb{O}}
\newcommand{\bbP}{\mathbb{P}}
\newcommand{\bbQ}{\mathbb{Q}}
\newcommand{\bbR}{\mathbb{R}}
\newcommand{\bbS}{\mathbb{S}}
\newcommand{\bbT}{\mathbb{T}}
\newcommand{\bbU}{\mathbb{U}}
\newcommand{\bbV}{\mathbb{V}}
\newcommand{\bbW}{\mathbb{W}}
\newcommand{\bbX}{\mathbb{X}}
\newcommand{\bbY}{\mathbb{Y}}
\newcommand{\bbZ}{\mathbb{Z}}

%% script math capitals
\newcommand{\sA}{\mathscr{A}}
\newcommand{\sB}{\mathscr{B}}
\newcommand{\sC}{\mathscr{C}}
\newcommand{\sD}{\mathscr{D}}
\newcommand{\sE}{\mathscr{E}}
\newcommand{\sF}{\mathscr{F}}
\newcommand{\sG}{\mathscr{G}}
\newcommand{\sH}{\mathscr{H}}
\newcommand{\sI}{\mathscr{I}}
\newcommand{\sJ}{\mathscr{J}}
\newcommand{\sK}{\mathscr{K}}
\newcommand{\sL}{\mathscr{L}}
\newcommand{\sM}{\mathscr{M}}
\newcommand{\sN}{\mathscr{N}}
\newcommand{\sO}{\mathscr{O}}
\newcommand{\sP}{\mathscr{P}}
\newcommand{\sQ}{\mathscr{Q}}
\newcommand{\sR}{\mathscr{R}}
\newcommand{\sS}{\mathscr{S}}
\newcommand{\sT}{\mathscr{T}}
\newcommand{\sU}{\mathscr{U}}
\newcommand{\sV}{\mathscr{V}}
\newcommand{\sW}{\mathscr{W}}
\newcommand{\sX}{\mathscr{X}}
\newcommand{\sY}{\mathscr{Y}}
\newcommand{\sZ}{\mathscr{Z}}


\renewcommand{\phi}{\varphi}

\renewcommand{\emptyset}{\O}

\providecommand{\abs}[1]{\lvert #1 \rvert}
\providecommand{\norm}[1]{\lVert #1 \rVert}


\providecommand{\x}{\times}




\providecommand{\ar}{\rightarrow}
\providecommand{\arr}{\longrightarrow}



\newcommand{\head}[1]{
	\begin{center}
		{\large #1}
		\vspace{.2 in}
	\end{center}
	
	\bigskip 
}



%----------Theorems----------

\newtheorem{theorem}{Theorem}[section]
\newtheorem{proposition}[theorem]{Proposition}
\newtheorem{lemma}[theorem]{Lemma}
\newtheorem{corollary}[theorem]{Corollary}

\theoremstyle{definition}
\newtheorem{definition}[theorem]{Definition}
\newtheorem*{definition*}{Definition}
\newtheorem{nondefinition}[theorem]{Non-Definition}
\newtheorem{exercise}[theorem]{Exercise}



\numberwithin{equation}{subsection}


%----------Title-------------
\newcommand{\hide}[1]{{\color{red} #1}} 
\newcommand{\com}[1]{{\color{blue} #1}} 
\newcommand{\meta}[1]{{\color{green} #1}} 

\begin{document}

\pagestyle{plain}


%%---  sheet number for theorem counter
\setcounter{section}{1}   

\head{MATH 161, Autumn 2024\\ SCRIPT 1: Sets, Functions and Cardinality } 




Sets and functions are among the most fundamental objects in mathematics.  A formal treatment
of set theory was first undertaken at the end of the 19th Century and was finally codified
in the form of the Zermelo-Fraenkel axioms.  While fascinating in its own right, pursuit of these
formalisms at this point would distract us from our main purpose of studying Calculus.  Thus, we
present a simplified version that will suffice for our immediate purposes.



\subsection*{Sets}

\begin{definition} (Working Definition)
A {\em set} is an object $S$ with the property that, given any $x$, we have the dichotomy that precisely
one of the two conditions $x\in S$ or $x\not\in S$ is true.  In the former case, we say that $x$ is an 
{\em element} of $S$, and in the latter, we say that $x$ is not an element of $S$.
\end{definition}


A set is often presented in one of the following forms:
\begin{itemize}
\item
A complete listing of its elements.

Example:  the set $S=\{1,2,3,4,5\}$ contains precisely the 
five smallest positive integers.


\item
A listing of some of its elements with ellipses to indicate unnamed elements.

Example 1:  the set $S=\{3, 4, 5, \ldots, 100\}$ contains the positive integers from 3 to 100,
including 6 through 99, even though these latter are not explicitly named.  


Example 2:  the set $S=\{2, 4, 6, \ldots, 2n, \ldots \}$ is the set of all positive even integers.


\item
A two-part indication of the elements of the set by first identifying the source of all elements
and then giving additional conditions for membership in the set.

Example 1: 
$S=\{x\in {\mathbb N}\mid \mbox{$x$ is prime}\}$ is the set of primes.  


Example 2:
$S=\{x\in {\mathbb Z}\mid \mbox{$x^2<3$}\}$ is the set of integers whose squares are less than 3.
\end{itemize}

\begin{definition}  
Two sets $A$ and $B$ are {\em equal} if they contain precisely the same elements, that is, $x\in A$
if and only if $x\in B$.  When $A$ and $B$ are equal, we denote this by $A=B$.
\end{definition}

\begin{definition}  
A set $A$ is a {\em subset} of a set $B$ if every element of $A$ is also an element of $B$, that is,
if $x\in A$, then $x\in B$.  When $A$ is a subset of $B$, we denote this by $A\subset B$.  If $A\subset B$ but $A\neq B$ 
we say that $A$ is a {\em proper} subset of $B.$ 
\end{definition}


\begin{exercise}
Let $A=\{1, \{2\}\}$.  
\\ Is $1\in A$?  Yes.
\\Is $2\in A$?  No.
\\Is $\{1\}\subset A$?  Yes.
\\Is $\{2\}\subset A$?  No.
\\Is $1\subset A$?  No.
\\Is $\{1\}\in A$?  No.
\\Is $\{2\}\in A$?  Yes.
\\Is $\{\{2\}\}\subset A$? Yes.  
\\Explain.
\end{exercise}

\begin{definition}  Let $A$ and $B$ be two sets. 
The \emph{union} of $A$ and $B$ is the set
\[
A \cup B = \{x \mid \text{$x \in A$ or $x \in B$} \}.
\]
\end{definition}

\begin{definition}  Let $A$ and $B$ be two sets. 
The \emph{intersection} of $A$ and $B$ is the set
\[
A \cap B = \{ x \mid \text{$x \in A$ and $x \in B$} \}.
\]
\end{definition}

\begin{theorem} \label{basicsets}
Let $A$ and $B$ be two sets.  Then:

\begin{enumerate}
\item[a)]
$A=B$ if and only if $A\subset B$ and $B\subset A$. 

\begin{proof}

Let $P(n)$ be the contrapositive of Theorem 1.7 (a), where we want to prove $P(n)$. Then, $P(n)$ is the statement that if $A \not \subset B$ or $B \not \subset A$, then $A \neq B$ (the or here is inclusive).

By Definition 1.3, if $A \not \subset B$, then there exists some $x \in A$ such that $x \notin B$. Therefore, by Definition 1.2, $A \neq B$.

Similarly, by Definition 1.3, if $B \not \subset A$, then there exists some $y \in B$ such that $y \notin A$. Therefore, by Definition 1.2, $A \neq B$.

This completes the proof.

\renewcommand\qedsymbol{QED}

\end{proof}

\item[b)]
$A\subset A\cup B$.

\begin{proof}

We prove this by contradiction. Assume $A \not \subset A \cup B$. Then, by Definition 1.3, there must exist some $x \in A$ such that $x \notin A \cup B$. This violates Definition 1.5. This completes the proof.

\renewcommand\qedsymbol{QED}

\end{proof}

\item[c)]
$A\cap B\subset A$.

\begin{proof}
We prove this by contradiction. Assume $A \cap B \not \subset A$. Then, by Definition 1.3, there must exist some $x \in A \cap B$ such that $x \notin A$. This violates Definition 1.6. This completes the proof.

\renewcommand\qedsymbol{QED}

\end{proof}

\end{enumerate}
\end{theorem}

A special example of the intersection of two sets is when the two sets have no elements in common.
This motivates the following definition.

\begin{definition}  
The \emph{empty set} is the set with no elements, and it is denoted $\emptyset$.  That is,
no matter what $x$ is, we have $x\not\in \emptyset$.
\end{definition}  

\begin{definition}  
Two sets $A$ and $B$ are \emph{disjoint} if $A\cap B=\emptyset$.
\end{definition}  

\begin{exercise}  
Show that if $A$ is any set, then $\emptyset\subset A$.

\begin{proof}
We prove this by contradiction. Assume $\emptyset \not\subset A$. Then, by Definition 1.3, there must exist some $x \in \emptyset$ such that $x \notin A$. This requires that $x \in \emptyset$, which violates Definition 1.8. This completes the proof.

\renewcommand\qedsymbol{QED}

\end{proof}

\end{exercise}


\begin{definition}  
Let $A$ and $B$ be two sets. 
The \emph{difference} of $B$ from $A$ is the set
\[
A \setminus B = \{ x \in A \mid x \notin B \}.
\]
\end{definition}

The set $A \setminus B$ is also called the \emph{complement} of $B$ relative to $A$.
When the set $A$ is clear from the context, this set is sometimes denoted $B^{c}$, but we will 
try to avoid this imprecise formulation and use it only with warning.

\begin{exercise} 
Let $A=\{x\in\bbN\mid x\text{ is even}\}; B=\{x\in\bbN\mid x\text{ is odd}\}; C=\{x\in\bbN\mid x\text{ is prime}\}; D=\{x\in\bbN\mid x\text{ is a perfect square}\}.$
Find all possible set differences.
\\$A \setminus B = A$.
\\$B \setminus A = B$.
\\$A \setminus C = \{x\in\bbN\mid x\text{ is even} \mid x\geq 4\}$.
\\$C \setminus A = \{x\in\bbN\mid x\text{ is prime} \mid x \neq 2\}$.
\\$A \setminus D = \{x\in\bbN\mid x\text{ is even} \mid \text{for all y $\in A$, }x \neq y^2\}$.
\\$D \setminus A = \{x\in\bbN\mid x\text{ is a perfect square} \mid \text{$\sqrt{x}\notin A$}\}$.
\\$B \setminus C = \{x\in B \mid x\text{ is not prime\}}$.
\\$C \setminus B =\{2\} $.
\\$B \setminus D = \{x\in B \mid x\text{ is not a perfect square\}}$.
\\$D \setminus B = \{x\in D \mid x\text{ is even\}}$.
\\$C \setminus D = C$.
\\$D \setminus C = D$.
\\$A \setminus A = B \setminus B = C \setminus C = D \setminus D = \emptyset$.

\end{exercise} 

\begin{theorem} 
Let $A,B$ and $X$ be sets.  Then:
\begin{enumerate}
\item[a)]
$X\setminus (A\cup B)=(X\setminus A)\cap (X\setminus B)$

\begin{proof}
We proceed by direct proof. By Definition 1.5 and Definition 1.11, the LHS is the set of all elements in $X$ that are in neither $A$ nor $B$. In formal notation, $X\setminus (A\cup B) = \{x \in X \mid x \notin A, x \notin B\}$.

By Definition 1.6 and Definition 1.11, the RHS is the set of all elements that are in both, $(X \setminus A)$ and $(X \setminus B)$. In turn, $(X \setminus A)$ is $\{x \in X \mid x \notin A\}$ and $(X \setminus B)$ is $\{x \in X \mid x \notin B\}$. Therefore, $X\setminus (A\cup B) = \{x \in X \mid x \notin A, x \notin B\}$.

Hence, the expression for LHS matches the expression for RHS. This completes the proof.


\renewcommand\qedsymbol{QED}

\end{proof}

\item[b)]
$X\setminus (A\cap B)=(X\setminus A)\cup (X\setminus B)$

\begin{proof}

We proceed by direct proof. By Definition 1.6 and Definition 1.11, the LHS is the set of all elements in $X$ that are in not in the intersection of $A$ and $B$, i.e., elements that are in $X$, and also are in either $A$ or $B$ or neither, but not both. In formal notation, this is $\{x \in X \mid x \notin A\cap B\}$.

By Definition 1.5 and Definition 1.11, the RHS is the set of all elements that are in $(X \setminus A)$ or $(X \setminus B)$ or both. This, in turn, is the set of all elements $(x \in X \mid x \notin A)$ or $(x \in X \mid x \notin B)$ or $(x \in X \mid x \notin A, x \notin B)$. If an element is not in $A$, or it is not in $B$, or it is in neither $A$ nor $B$, then it is not in $A \cap B$. Therefore, the RHS is $\{x \in X \mid x \notin A\cap B\}$.

Hence, the expression for LHS matches the expression for RHS. This completes the proof.


\renewcommand\qedsymbol{QED}

\end{proof}

\end{enumerate}
\end{theorem}

Sometimes we will encounter families of sets. The definitions of intersection/union can be extended to infinitely many sets. 


\begin{definition}

 Let $\mathcal{A}=\{A_\lambda\mid \lambda\in I\}$ be a collection of sets indexed by a nonempty set $I.$ Then the intersection and union of $\mathcal{A}$ are the sets
$$\bigcap_{\lambda\in I} A_\lambda =\{x\mid x\in A_\lambda, \text{ for all } \lambda\in I\},$$
and
$$\bigcup_{\lambda\in I}A_\lambda =\{x\mid x\in A_\lambda, \text{ for some }\lambda\in I\}.$$
\end{definition}

\begin{theorem}  
Let $X$ be a set, and let  $\mathcal{A}=\{A_\lambda\mid \lambda\in I\}$ be a nonempty collection of sets. Then:
\begin{enumerate}
\item
$X\setminus \left( \bigcup_{\lambda\in I}A_\lambda\right) =\bigcap_{\lambda\in I} (X\setminus A_\lambda)$.

\begin{proof}
We proceed by direct proof. By Definition 1.11 and Definition 1.14, the LHS is the set $\{x \in X \mid x \notin \bigcup_{\lambda\in I}A_\lambda\}$. Since $x \notin \bigcup_{\lambda\in I}A_\lambda\ $, this implies that for all $\lambda \in I$, $x \notin A_\lambda$.

By Definition 1.11 and Definition 1.14, the RHS is the intersection of sets $\{x \in X \mid x \notin A_\lambda \}$ for all $\lambda \in I$.

Hence, the expression for LHS matches the expression for RHS. This completes the proof.

\renewcommand\qedsymbol{QED}
\end{proof}

\item
$X\setminus \left( \bigcap_{\lambda\in I}A_\lambda\right)  =\bigcup_{\lambda\in I} (X\setminus A_\lambda).$

\begin{proof}
We proceed by direct proof. By Definition 1.11 and Definition 1.14, the LHS is the set $\{x \in X \mid x \notin \bigcap_{\lambda\in I}A_\lambda\}$. Since $x \notin \bigcap_{\lambda\in I}A_\lambda$, this implies that $\neg \forall \lambda \in I: x \in A_\lambda$. This is equivalent to $\exists \lambda \in I: x \in X, x \notin A_\lambda$.

By Definition 1.11 and Definition 1.14, the RHS is the set $\{x \mid x \in X \setminus A_\lambda, \text{for some } \lambda \in I \}$. This is equivalent to $\exists \lambda \in I : x \in X, x \notin A_\lambda$.

Hence, the expression for LHS matches the expression for RHS. This completes the proof.

\renewcommand\qedsymbol{QED}
\end{proof}
\end{enumerate}
\end{theorem}\medskip

\begin{definition}  Let $A$ and $B$ be two nonempty sets. 
The \emph{Cartesian product} of $A$ and $B$ is the set of ordered pairs
\[
A \times B = \{ (a, b) \mid \text{$a \in A$ and $b \in B$} \}.
\]
If $(a, b)$ and $(a', b') \in A \times B$, we say that $(a, b)$ and $(a', b')$ are
\emph{equal} if and only if $a = a'$ and $b = b'$. In this case, we write $
(a, b) = (a', b').$


\end{definition}




\subsection*{Functions}

\begin{definition} Let $A$ and $B$ be two nonempty sets.  
A \emph{function} $f$ from $A$ to $B$ is a subset $f \subset A \times B$ such that for all $a \in A$ there exists a unique $b \in B$ satisfying $(a, b) \in f$.  To express the idea that $(a, b) \in f$, we most
often write $f(a) = b$.  To express that $f$ is a function from $A$ to $B$ in symbols we write $f \colon A \rightarrow B$.  
\end{definition}


\begin{exercise}  
Let the function $f \colon \mathbb{Z} \rightarrow \mathbb{Z}$ be defined by
$f(n)=2n$.  Write $f$ as a subset of $\mathbb{Z} \times \mathbb{Z}$.  

$f \subset \bbZ \times \bbZ$ is the set $f = \{\left(n, 2n\right) \mid n \in \bbZ\} = \{\dots, (-2, -4), (-1, -2), (0,0), (1,2), (2,4), \dots\}$.

\end{exercise}

\begin{definition}  Let $f \colon A \rightarrow B$ be a function.  The \emph{domain} of $f$ is $A$ and the \emph{codomain} of $f$ is $B.$\\
If $X \subset A$, then the \emph{image of $X$ under $f$} is the set
\[
f(X) = \{ f(x) \in B \mid  x \in X \}.
\]
If $Y \subset B$, then the \emph{preimage of $Y$ under $f$} is the set
\[
f^{-1}(Y) = \{ a \in A \mid f(a) \in Y \}.
\]
\end{definition}

\begin{exercise}
Must $f(f^{-1}(Y))=Y$ and $f^{-1}(f(X))=X?$ For each, either prove that it always holds or give a counterexample.

Let $A$ and $B$ be non-empty sets such that $A=\{1,2\}, B =\{1,2,3\}$. Let $f \colon A \rightarrow B$ be defined by $f(1) = 1, f(2)=2$. Let $Y \subset B=\{2,3\}$. Then $f^{-1}(Y)=\{2\}$ and $f(f^{-1}(Y))=\{2\}\not=Y$. Thus, we have a counterexample where $f(f^{-1}(Y))\not=Y$.

Let $A$ and $B$ be non-empty sets such that $A=\{1,2,3\}, B =\{1,2\}$. Let $f \colon A \rightarrow B$ be defined by $f(1) = 1, f(2)=2, f(3)=2$. Let $X \subset A=\{2\}$. Then $f(X)=\{2\}$ and $f^{-1}(f(X))=\{2,3\}\not=X$. Thus, we have a counterexample where $f^{-1}(f(X))\not=X$.

\end{exercise}


\begin{definition}  A function $f \colon A \rightarrow B$ is \emph{surjective} (also known as `onto') if, 
for every $b\in B$, there is some $a\in A$ such that $f(a) = b$.  The function $f$ is \emph{injective} (also known as `one-to-one') if for all $a, a' \in A$, if $f(a) = f(a')$, then $a = a'$.  The function $f$ is \emph{bijective}, (also known as a bijection or a `one-to-one' correspondence) if it is surjective and injective.
\end{definition}



\begin{exercise}
Let $f:{\mathbb N}\rightarrow {\mathbb N}$ be defined by $f(n)=n+2$.  Is $f$ injective? Yes, since $a+2 = a'+2$ implies $a = a'$.  Is $f$ surjective? No, since 1 and 2 are not in the image of $f$.
\end{exercise}

\begin{exercise}
Let $f:{\mathbb Z}\rightarrow {\mathbb Z}$ be defined by $f(x)=x+2$.  Is $f$ injective? Yes (same proof as above, for Exercise 1.22).  Is $f$ surjective? Yes, since for all $y \in \bbZ$, there exists $x \in \bbZ$ such that $y=x+2$.
\end{exercise}

\begin{exercise}
Let $f:{\mathbb N}\rightarrow {\mathbb N}$ be defined by $f(n)=n^2$.  Is $f$ injective? Yes, since $a^2 = (a')^2$ implies $a = a'$.  Is $f$ surjective? No, since 2 is not in the image of $f$.
\end{exercise}

\begin{exercise}
Let $f:{\mathbb Z}\rightarrow {\mathbb Z}$ be defined by $f(x)=x^2$.  Is $f$ injective? No, since $a^2 = (a')^2$ does not imply $a=a'$. This is because $a=-a'$ also leads to $a^2 = (a')^2$.  Is $f$ surjective? No, since 2 is not in the image of $f$.
\end{exercise}




\begin{definition}
Let $f:A\longrightarrow B$ and $g:B\longrightarrow C. $ Then the \emph{composition} $g\circ f: A\longrightarrow C$ is defined by $(g\circ f)(x)=g(f(x)),$ for all $x\in A.$ 
\end{definition}

\begin{proposition}  Let $A$, $B$, and $C$ be sets and suppose that $f:A\longrightarrow B$  and  $g:B\longrightarrow C.$  Then $g\circ f:A\longrightarrow C$ and
\begin{enumerate}
\item[a)] if $f$ and $g$ are both injections, so is $g\circ f.$
\begin{proof}
We prove this by contradiction. Assume that $g \circ f$ is not an injection, i.e., $(g \circ f)(a)=(g \circ f)(a')$ but $a \not = a'$. 

Since $g$ is an injection, we know that $b = b'$, i.e., $f(a) = f(a')$. Since $f$ is an injection, this implies that $a = a'$. We have reached a contradiction and this completes the proof.

\renewcommand\qedsymbol{QED}
\end{proof}
\item[b)] if $f$ and $g$ are both surjections, so is $g\circ f.$
\begin{proof}
We prove this by contradiction. Assume that $g \circ f$ is not a surjection, i.e., $\exists c \in C$ such that $\forall a \in A: (g \circ f)(a) \not = c$. In other words, we assume that there exists some $c \in C$ such that there is no $a \in A$ for which $(g \circ f)(a)=c$. Let this value of $c$ be denoted by $c_0$.

Since $g$ is surjective, $\forall c\in C: \exists b \in B$ such that $g(b) = c$. Hence, let $b_0$ be defined by $g(b_0)=c_0$. Since $f$ is surjective, $\forall b\in B: \exists a \in A$ such that $f(a) = b$. Hence, let $a_0$ be defined by $f(a_0)=b_0$.

Then, $(g \circ f)(a_0)=g((f(a_0'))=g(b_0)=c_0$, and we have reached a contradiction. This completes the proof. 

\renewcommand\qedsymbol{QED}
\end{proof}
\item[c)] if $f$ and $g$ are both bijections, so is $g\circ f.$
\begin{proof}

We have shown above that if $f$ and $g$ are both injections, so is $g\circ f$, and if $f$ and $g$ are both surjections, so is $g\circ f$.

Since $f$ and $g$ are both bijections, i.e., $f$ and $g$ are both injections as well as surjections, we know that $g\circ f$ is both an injection and a surjection. Therefore, $g\circ f$ is a bijection.

\renewcommand\qedsymbol{QED}
\end{proof}
\end{enumerate}
\end{proposition} 

\begin{proposition} 
Suppose that $f \colon A \rightarrow B$ is bijective.  
Then there exists a bijection $g \colon B \rightarrow A$ that satisfies $(g\circ f)(a)=a, \forall a\in A$, and $(f\circ g)(b)=b,$ for all $b\in B.$ 
The function $g$ is often called the \emph{inverse} of $f$ and  denoted $f^{-1}$. It should not be confused with the preimage. 
\begin{proof}

We know that $f$ is surjective, i.e., for all $b \in B$, there exists some $a_b \in A$ such that $f(a_b)=b$.

We use this formulation to define $g(b)=a_b$. Then $(f\circ g)(b)=f(g(b))=f(a_b)=b$ for all $b \in B$. 

Let $a \in A$ and let $b =f(a)$. We know that $b = f(a_b)$, hence $f(a)=f(a_b)$ which implies $a = a_b$, since $f$ is injective.

Therefore, $(g \circ f)(a)=g(b)=a_b=a$, hence $(g \circ f)(a)=a$ for all $a \in A$.

For all $a \in A$, there exists $b \in B$ such that $g(b)=a$, namely $g(b)=a_b$. This follows from our earlier formulation of $a_b$. Hence, $g$ is surjective.

Suppose $g(b)=g(b')$. Then, it follows that $a_b=a_{b'}$. We know that $a_b=a_{b'}$ implies $f(a_b)=f(a_{b'})$, i.e., $b=b'$. Hence, $b=b'$ follows from $g(b)=g(b')$, and $g$ is injective.

Since $g$ is both surjective and injective, we know that it is bijective. This completes the proof.


\renewcommand\qedsymbol{QED}
\end{proof}

\end{proposition}



\begin{definition}
We say that two sets $A$ and $B$ are in \emph{bijective correspondence} when there exists a bijection from $A$ to $B$ or, equivalently, from $B$ to $A$.
\end{definition}




\subsection*{Cardinality}

\begin{definition}  
Let $n \in \mathbb{N}$ be a natural number.  We define $[n]$ to be the set $\{1, 2, \dotsc, n \}$.  
Additionally, we define $[0]=\emptyset$.
\end{definition}

\begin{definition}  
A set $A$ is \emph{finite} if $A=\emptyset$ or if there exists a natural number $n$ and a bijective correspondence between $A$ and the set $[n].$   If $A$ is not finite, we say that $A$ is \emph{infinite}.
\end{definition}




\begin{theorem}

Let $n, m\in {\mathbb N}$ with $n<m$.  \\ Then there does not exist an injective function
$f:[m]\rightarrow [n]$.
\end{theorem}
{\it Hint: Fix $k\in\mathbb N.$ Prove, by induction on $n$, that for all $n\in\mathbb{N},$ there is no injective function $f:[n+k]\rightarrow [n].$} 

\begin{proof}[Proof by Contradiction]
Since $n, m \in \bbN$ and $m>n$, let $m=n+k$ for some $k \in \bbN$. We want to prove, by induction on $n$, that for all $n\in\mathbb{N},$ there is no injective function $f:[n+k]\rightarrow [n].$

Consider the base case for $n=1$. We want to show that there is no injective function $f:[1+k]\rightarrow [1].$ We know that $[1+k] = \{1, 2, \dots, 1+k\}$ and $[1] = \{1\}$. Since $k \in \bbN$, we have that $k>0$, i.e., $k \geq 1$. Since there is only one element in the set $[1]$, any function $f\colon [1+k] \rightarrow [1]$ must be defined by $f(x)=1$ for all $x \in [1+k]$. 

Hence, by this definition, for all $x \in [1+k], f(x)=f(x')=1$ even if $x \not = x'$. We have an explicit counterexample, since $1,2 \in [1+k]$ and $f(1)=1=f(2)$. Therefore, there is no injective function $f:[1+k]\rightarrow [1].$ This completes the base case.

Our inductive step is that we assume there is no injective function $g':[n+k]\rightarrow [n]$, then we use this to show that there is no injective function $g:[n+k+1]\rightarrow [n+1].$ By contraposition, this is equivalent to saying that if there exists an injective function $g:[n+k+1]\rightarrow [n+1]$, then there exists an injective function $g':[n+k]\rightarrow [n].$ 

Hence, we begin by assuming that there exists an injective function $g:[n+k+1]\rightarrow [n+1]$.

Case 1: $\not \exists a \in [n+k+1]$ such that $g(a)=n+1$. Then, define $g'\colon [n+k] \rightarrow [n]$ by $g'(x)=g(x)$. Then, if $g$ is injective, so is $g'$.

Case 2: $\exists! a \in [n+k+1]$ such that $g(a)=n+1$. (Note: we know that any such $a$ must be unique from our assumption that $g$ is injective). Then, for all $x \in [n+k]$, define 
\[
g'(x) =
\begin{cases}
g(x), \text{if } x < a \\
g(x+1), \text{if } x \geq a
\end{cases}
\]

We first show that for all $x \in [n+k], g'(x) \in [n]$. 

Case un: If $x < a$, i.e., $x \not= a$, then since $g$ is injective, $g(x) \not= g(a) = n+1$, hence $g'(x)=g(x) \in [n]$. 

Case deux: If $x \geq a$, i.e, $x+1 > a$ and $x+1 \not= a$, then since $g$ is injective, $g(x+1) \not= g(a) = n+1$, hence $g'(x)=g(x+1) \in [n]$.

We now show that $g'(x)$ is injective. To do this, we take two elements $p, q \in [n+k]$ such that $p \not = q$, and we show that $g'(p) \not= g'(q)$. 

Case one: $p, q < a$. In this case, $g'(p)=g(p)$ and $g'(q)=g(q)$. Since $g$ is injective, if $p \not= q$, then $g(p) \not= g(q)$, hence $g'(p) \not= g'(q)$. Therefore, $g'$ is injective.

Case two: $p < a$ and $q \geq a$. From these conditions, it follows that $q+1 > a > p$, hence $p \not= q+1$. In this case, $g'(p)=g(p)$ and $g'(q)=g(q+1)$. Since $g$ is injective, if $p \not= q+1$, then $g(p) \not= g(q+1)$, hence $g'(p) \not= g'(q)$. Therefore, $g'$ is injective.

Case three: $q < a$ and $p \geq a$. From these conditions, it follows that $p+1 > a > q$, hence $p+1 \not= q$. In this case, $g'(p)=g(p+1)$ and $g'(q)=g(q)$. Since $g$ is injective, if $p+1 \not= q$, then $g(p+1) \not= g(q)$, hence $g'(p) \not= g'(q)$. Therefore, $g'$ is injective.

Case four: $p, q \geq a$. In this case, $g'(p)=g(p+1)$ and $g'(q)=g(q+1)$. Since $g$ is injective, if $p \not= q$, i.e., $p+1 \not= q+1$, then $g(p+1) \not= g(q+1)$, hence $g'(p) \not= g'(q)$. Therefore, $g'$ is injective.

This completes the proof.

\renewcommand\qedsymbol{QED}
\end{proof}

The above proof is the one we did in class. I have also attempted another proof below, which I am not entirely sure is valid.

\begin{proof}[Direct Proof by Induction]
Since $n, m \in \bbN$ and $m>n$, let $m=n+k$ for some $k \in \bbN$. We want to prove, by induction on $n$, that for all $n\in\mathbb{N},$ there is no injective function $f:[n+k]\rightarrow [n].$

Consider the base case for $n=1$. We want to show that there is no injective function $f:[1+k]\rightarrow [1].$ We know that $[1+k] = \{1, 2, \dots, 1+k\}$ and $[1] = \{1\}$. Since $k \in \bbN$, we have that $k>0$, i.e., $k \geq 1$. Since there is only one element in the set $[1]$, any function $f\colon [1+k] \rightarrow [1]$ must be defined by $f(x)=1$ for all $x \in [1+k]$. 

Hence, by this definition, for all $x \in [1+k], f(x)=f(x')=1$ even if $x \not = x'$. We have an explicit counterexample, since $1,2 \in [1+k]$ and $f(1)=1=f(2)$. Therefore, there is no injective function $f:[1+k]\rightarrow [1].$ This completes the base case.

Next, we want to show that if there is no injective function $f:[n+k]\rightarrow [n]$, then there is no injective function $f:[n+k+1]\rightarrow [n+1].$ 

Since there is no injective function $f:\{1, 2, \dots, n+k\}\rightarrow \{1, 2, \dots, n\}$, this implies that for any function $f:\{1, 2, \dots, n+k\}\rightarrow \{1, 2, \dots, n\}$, there exist some $a, a' \in [n+k]$ such that $f(a)=f(a')$ but $a \not = a'$.

$[n+k] \subset [n+k+1]$ and $[n] \subset [n+1]$, hence for all $y \in [n+k], y \in [n+k+1]$ and for all $z \in [n], z \in [n+1]$.

Therefore, for any function $f:\{1, 2, \dots, n+k+1\}\rightarrow \{1, 2, \dots, n+1\}$, we know that there exist $a, a' \in [n+k+1]$ such that $f(a)=f(a')$ but $a \not = a'$, and this completes the inductive step.

This completes the proof.

\renewcommand\qedsymbol{QED}

\end{proof}


\begin{theorem}  \label{bij}
Let $A$ be a finite set. Suppose that $A$ is in bijective correspondence both with $[m]$ and with $[n]$.  Then $m = n$.

\begin{proof}
We prove this by contradiction. Assume $m \not = n$, and without loss of generality let $m = n + k$, where $k \in \bbN$.

Since A is in bijective correspondence with $[m]$, there exists a bijective function $f\colon A \rightarrow [m]$. Hence, there exists an inverse bijective function $f^{-1}\colon [m] \rightarrow A$. 

Since A is in bijective correspondence with $[n]$, there exists a bijective function $g\colon A \rightarrow [n]$.

Then, $g \circ f^{-1}\colon [m] \rightarrow [n]$ must be a bijection. We know that $m = n+k$, hence this implies that there exists a bijection from $[n+k]$ to $[n]$. 

We know that any function from $[n+k]$ to $[n]$ cannot be injective, hence there cannot be such a bijection. We have reached a contradiction, and this completes the proof.


\renewcommand\qedsymbol{QED}

\end{proof}

\end{theorem}

The preceding result allows us to make the following important definition.

\begin{definition}[Cardinality of a finite set]
 If $A$ is a finite set that is in bijective correspondence with $[n]$, then we say that the \emph{cardinality} of $A$ is $n$, and we write $\abs{A} = n$.  (By   Theorem~\ref{bij}, there is exactly one such natural number $n$.) We define the cardinality of the empty set to be $0.$
\end{definition}





\begin{exercise}
Let $A$ and $B$ be finite sets. 
\begin{enumerate}
\item[a)]
 If $A\subset B$, then $|A|\leq |B|$.
\begin{proof}
Suppose $A$ is in bijective correspondence with $[p]$ and $B$ is in bijective correspondence with $[q]$.

Let $f\colon [p] \rightarrow A$ be an injection, and let $g \colon B \rightarrow [q]$ be an injection. Let $h \colon A \rightarrow B$ be defined by $h(x)=x$ for all $x \in A$. Hence, if $h(x)=h(x')$, then $x=x'$, so $h$ is an injection.

Theorem 1.32 says that if $m > n$, then there does not exist an injection from $[m]$ to $[n]$. The contrapositive is that if there exists an injection from $[m]$ to $[n]$, then $m \leq n$.

Since $h$ is an injection from $A$ to $B$, we know that $|A| \leq |B|$. This completes the proof.

\renewcommand\qedsymbol{QED}
\end{proof}
 
\item[b)]
Let  $A\cap B=\emptyset.$ Then $|A\cup B|=|A|+|B|.$ 
\begin{proof}

We want to show that if $A$ is in bijective correspondence with $[p]$ and $B$ is in bijective correspondence with $[q]$, then $A \cup B$ is in bijective correspondence with $[p+q]$.

If $A \cap B = \emptyset$, then for all $a \in A$, $a \notin B$. Hence $A \cup B = \{a_1, a_2, \dots, a_n, b_1, b_2, \dots, b_n\}$. 

We know that $[q] = \{1, 2, \dots, q\}$ is in bijective correspondence with $\{p+1, p+2, \dots, p+q\}$. 

We know that the elements $\{a_1, a_2, \dots, a_n\}$ are in bijective correspondence with $[p]$. The elements $\{b_1, b_2, \dots, b_n\}$ are in bijective correspondence with $[q]$, and hence also with $\{p+1, p+2, \dots, p+q\}$. 

Hence, $A \cup B$ is in bijective correspondence with $\{1, 2, \dots, p, p+1, p+2, \dots, p+q\}=[p+q]$.

\renewcommand\qedsymbol{QED}
\end{proof}
\item[c)] 
$|A\cup B|+|A\cap B|=|A|+|B|$.
\begin{proof}
We rewrite the expression in terms of unions of disjoint sets, so that we can use the result from part b. Thus, $A\cup B = (A\setminus B) \cup (B \setminus A) \cup (A\cap B)$.

$A = (A \setminus B) \cup (A \cap B)$. 

$B = (B \setminus A) \cup (A \cap B)$.

Then, the LHS is $|A\cup B|+|A\cap B| = |A\setminus B| + |B \setminus A| + 2\cdot |A\cap B|$.

The RHS is $|A|+|B|=|A\setminus B| + |B \setminus A| + 2\cdot |A\cap B|$. This completes the proof.


\renewcommand\qedsymbol{QED}
\end{proof}
\item[d)]  
 $|A\times B|=|A|\cdot |B|$.
 \begin{proof}
Let $n = |A|$. We prove this by induction on $n$. For the base case, let $n=0$, i.e., $A = \emptyset$. Then, since there are no elements in $A$, we know that $A \times B = \emptyset$. Hence, $|A\times B|=0=0 \cdot |B|=|A|\cdot |B|= n \cdot |B|$.

For the inductive step, we want to show that if $|A|=n+1$, then $|A\times B|=(n+1)\cdot |B|$. 

Let $A' = A \setminus \{x\}$ for some $x \in A$. Then, $A \times B = (A' \times B) \cup (\{x\} \times B)$. 

We know that $|A'|=n$, hence by the inductive hypothesis, $|A' \times B| = |A'| \cdot |B| = n \cdot |B|$. 

Next, $|\{x\} \times B| = |\{x\}| \cdot |B| = 1 \cdot |B|$. Since $A' \cap \{x\} = \emptyset$, we know that $|A \times B| = |A' \times B| + |\{x\} \times B| = n \cdot |B| + 1 \cdot |B| = (n+1) \cdot |B|$. This completes the proof.
 
 
\renewcommand\qedsymbol{QED}
\end{proof}
 \end{enumerate}
\end{exercise}




{\bf Important Note:  In future scripts you may assume basic properties of finite sets and methods of counting elements without having to refer back to the notions presented in this script.When $|A|=n,$ we say that $A$ contains $n$ elements.}

\begin{definition}
An set $A$ is said to be {\em countable} either if it is finite if or if it is in bijective correspondence with $\bbN.$ An infinite set that is not countable is called {\em uncountable}.
\end{definition}

\begin{exercise}
Prove that $\bbZ$ is a countable set.

\begin{proof}
To prove that $\bbZ$ is a countable set, we must show that there exists a bijection between $\bbZ$ and $\bbN$. To do this, we construct a function $f\colon \bbZ \rightarrow \bbN$ and then show that it is bijective.

Let this function $f$ be defined by
\[
f(x) = \begin{cases}
1 & \text{if } x = 0,\\
2x  & \text{if } x>0,\\
-2x+1  & \text{if } x<0.
\end{cases}
\]

First, we show that $f$ is surjective. We know that $f(0)=1$, so we need to show that $f(x)$ reaches every $n \in \bbN, n>1$. All natural numbers greater than 1 are either even or odd. For all even natural numbers, we have that $n = f(\frac{n}{2})$. For all odd natural numbers, we have that $n = f(\frac{1-n}{2})$.

Next, we show that $f$ is injective. By contraposition, we want to show that if $x \not= x'$, then $f(x) \not= f(x')$. We know that 0 is the only element in $\bbZ$ such that $f(x)=1$. We also know that if $x>0$ and $x'<0$ or vice versa, then $f(x) \not= f(x')$. Finally, if $x, x' >0$ and $x \not=x'$, then $2x\not=2x'$. Similarly, if $x, x' <0$ and $x \not=x'$, then $-2x+1\not=-2x'+1$.

This completes the proof.

\renewcommand\qedsymbol{QED}
\end{proof}

\end{exercise}


\begin{theorem} \label{subsetbbN}
  Every subset of $\bbN$ is countable.
  
Hint: when $A$ is an infinite subset of $\bbN$,
    construct a bijection $f\colon\bbN\to A$ inductively/recursively. This looks similar to a proof by induction: define an initial term (or terms)
explicitly and then present a rule that defines $f(n+1)$ assuming that
$f(1),\ldots, f(n) $ have already been defined. For example, the factorial of $n$ is defined
inductively by letting $0! = 1$ and
\begin{align*}
  (n+1)! & = (n+1)\cdot n!
\end{align*}
for $n\geq 0$. After constructing your function you must verify that it is indeed a bijection.

\begin{proof}
We prove this by contradiction. Assume there is some set $M \subset \bbN$ that is not countable. 

Case 1: $M$ is finite. In this case, let $q=|M|$. Then, we know that there is a bijective correspondence between $M$ and $[q]$. Hence, $M$ is countable and we have reached a contradiction.

Case 2: $M$ is infinite, hence $M \not = \emptyset$. By the well-ordering principle, we know that every non-empty subset of $\bbN$ has a least element. Therefore, for any $K \subset \bbN$, let $j(K)$ denote the least element of $K$. 

Then, let $f \colon \bbN \rightarrow M$ be defined by $f(1)=j(M)$ and for all $n\in \bbN, n\geq1$, 
\[
f(n+1)=j(M\setminus\{f(1),f(2),\dots,f(n)\}).
\]

We want to show that $f$ is bijective. We know that for all $m \in M$, there exists some subset $M_m \subset M$ such that $m=j(M_m)$. Then, we want to show that for all $m \in M$, there exists a unique $n \in \bbN$ such that $f(n)=m$. We demonstrate this by construction: for all $m \in M, n = 1 + |M\setminus M_m|$ is the unique element in $\bbN$ such that $f(n)=m$. 

First, we show that $f(1 + |M\setminus M_m|) =m$. By the recursive definition of $f$, we know that $f(1 + |M\setminus M_m|)=j(M\setminus\{f(1),f(2),\dots,f(|M \setminus M_m|)\} = m$.

Next, we show that this $n$ is unique. Suppose there is some $n' \not= n$. Then, $f(n')=j(M\setminus\{f(1),f(2),\dots,f(n'-1)\}$. Since $n' \not= n, \{f(1),f(2),\dots,f(n'-1)\} \not = \{f(1),f(2),\dots,f(n-1)\}$, hence $M\setminus\{f(1),f(2),\dots,f(n'-1)\} \not= M\setminus\{f(1),f(2),\dots,f(n-1)\}$, hence $j(M\setminus\{f(1),f(2),\dots,f(n'-1)\}) \not= j(M\setminus\{f(1),f(2),\dots,f(n-1)\})$. Therefore, $f(n') \not= f(n)$.



This completes the proof.

\renewcommand\qedsymbol{QED}
\end{proof}

\end{theorem}

\begin{theorem}\label{injbbN}
  If there exists an injection $f:A\longrightarrow B$ where $B$ is countable, then $A$ is
  countable. {\it Hint: Use Theorem~\ref{subsetbbN}.}

\begin{proof}
Case 1: $f$ is bijective. In this case, $|A|=|B|$, so if $B$ is countable, then $A$ is countable.

Case 2: $f$ is not bijective, i.e., $f$ is injective but not surjective. Then, there exists some element $b_0 \in B$ such that for all $a \in A, f(a) \not = b_0$. Hence, $|A| < |B|$. 

Subcase one: If $B$ is finite, and $B$ is countable, then there exists a bijective correspondence from $B$ to $[k]$ for some $k \in \bbN$. Then, we know that there exists a bijective correspondence from $A$ to $[j]$ for some $j \in \bbN, j < k$. Hence, $A$ is countable.

Subcase two: If $B$ is infinite, and $B$ is countable, then there exists a bijective correspondence $g\colon B \rightarrow \bbN$. Since $f\colon A \rightarrow B$ is an injection, we have that $g \circ f \colon A \rightarrow X$ is a bijection, where $X \subset \bbN$. 

By Theorem 1.38, every subset of $\bbN$ is countable, hence $X$ is countable. $A$ is in bijective correspondence with $X$, therefore $A$ is countable. This completes the proof.

\renewcommand\qedsymbol{QED}
\end{proof}
\end{theorem}

\begin{corollary}
  Every subset of a countable set is also countable.

  

\begin{proof}
Let $B$ be a countable set, and let $A \subset B$. Then, for all $a \in A, a \in B$.

Let $f \colon A \rightarrow B$ be defined by $f(n)=n$. If $f(n)=f(n')$, then $n =n'$, hence $f$ is injective.

Since there exists an injection $f \colon A \rightarrow B$ and $B$ is countable, $A$ is also countable.



\renewcommand\qedsymbol{QED}
\end{proof}

\end{corollary}

\begin{corollary}
  If there exists a surjection $g\colon B\to A$ where $B$ is countable, then $A$ is countable. {\it
    Hint: Use Theorem~\ref{injbbN}.}

\begin{proof}
Since $B$ is countable, either there exists a bijective correspondence $f\colon B \rightarrow [n]$ or there exists a bijective correspondence $f\colon B \rightarrow \bbN$.

Since $g \colon B \rightarrow A$ is surjective, we know that for all $a \in A$, there exists $b_i \in B$ such that $g(b_i)=a$.

If $A$ is finite, then we define $h \colon A \rightarrow [n]$. If $A$ is infinite, then we define $h \colon A \rightarrow \bbN$. In both cases, for all $a \in A, h(a) = \text{min}\{i \in \bbN \mid g(b_i)=a\}$.

We want to show that $h$ is injective. Suppose $a, a' \in A$ are such that $h(a)=h(a')$. Then, $g(b_i)=a$ and $g(b_i)=a'$, so $a=a'$. Hence, $h$ is injective and since both $[n]$ and $\bbN$ are countable, therefore $A$ is countable.


\renewcommand\qedsymbol{QED}
\end{proof}
\end{corollary}

\begin{exercise}
Prove that $\bbN\times \bbN$ is countable by considering the function $f:\bbN\times\bbN\longrightarrow \bbN$ given by $f(n,m)=(10^n-1)10^m$.

{\em (Alternatively you could use either one of the functions $  g(n,m) = 2^n\cdot 3^m$ and 
  $h(n,m)  = \binom{n+m}{2} + n.$)}

\begin{proof}

Let $g \colon \bbN\times \bbN \rightarrow \bbN$ be defined by $g(n,m) = 2^n\cdot 3^m$. 

Assume $n \not = n_0$, and without loss of generality suppose $n_0 < n$. We want to show that $g(n_0,m) \not = g(n,m)$. 

Suppose $g(n_0,m) = g(n,m)$, i.e., $2^{n_0}\cdot 3^m = 2^{n}\cdot 3^m$. Then, dividing both sides by $2^{n_0}$, we get $3^m = 2^{n-n_0} \cdot 3^m$. Since $n \not= n_0, n-n_0 \not = 0$, hence the LHS is odd while the RHS is even. Thus, we have reached a contradiction.

Similarly, assume $m \not = m_0$, and without loss of generality suppose $m_0 < m$. We want to show that $g(n,m_0) \not = g(n,m)$.

Suppose $g(n,m_0) = g(n,m)$, i.e., $2^{n}\cdot 3^{m_0} = 2^{n}\cdot 3^m$. Then, dividing both sides by $3^{m_0}$, we get $2^n = 2^{n} \cdot 3^{m-m_0}$. Then, the RHS is divisible by 3 but the LHS is not. Thus, we have reached a contradiction.

Therefore, $g$ is injective. By Theorem 1.39, there exists an injection $g \colon \bbN\times \bbN \rightarrow \bbN$ and $\bbN$ is countable. Therefore, $\bbN\times \bbN$ is countable.

\renewcommand\qedsymbol{QED}
\end{proof}
\end{exercise}
\bigskip

\begin{center}
{\em Additional Exercises}
\end{center}

{\em In all exercises you are expected to prove your answer, unless explicitly stated otherwise.}


\begin{enumerate}


\item In each of the following, write out the elements of the sets.

\begin{enumerate}
\item[a)] $(\left\{n \in \bbZ\mid \text{n is divisible by 2}\right\} \cap \bbN) \cup \{-5\}$ 
\item[b)] $\left\{F,G,H\right\}\times\left\{5,8,9\right\}$ 
\item[c)] $\left\{[n] \mid n \in \bbN, 1 \leq n \leq 3 \right\}$ 
\item[d)]  $\left\{ (x,y) \in \bbN \times \bbN \mid y = 2x,\; x = 2y \right\}$ 
\item[e)] $(\{1,2\} \times \{1,2\}) \times \{1,2\}$
\item[f)] $(\{1,2\} \cup \{1,2\}) \cup \{1,2\}$
\item[g)] $\{1,2\}\cup\emptyset$
\item[h)] $\{1,2\}\cap\emptyset$
\item[i)] $\{1,2\}\cup\{\emptyset\}$
\item[j)] $\{1,2\}\cap\{\emptyset\}$
\item[k)] $\{\{a\}\cup\{b\}\mid a \in \bbN, b \in \bbN, 1 \leq a \leq 4, 3 \leq b \leq 5\}$
\item[l)] $\{\{12\} \} \cup \{12\}$
\end{enumerate} 


\item Let $A$ and $B$ be subsets of the set $X.$ 
The symmetric sum $A\oplus B$ (sometimes also called symmetric difference) of sets $A$ and $B$ is defined by
$$A\oplus B=(A\setminus B)\cup (B\setminus A).$$
Prove that 
$$A\oplus B=(A\cup B)\cap [X\setminus(A\cap B)].$$

\begin{proof}
We proceed by direct proof. By Theorem 1.7, $A\oplus B=(A\cup B)\cap [X\setminus(A\cap B)]$ if and only if $A\oplus B\subset(A\cup B)\cap [X\setminus(A\cap B)]$ and $(A\cup B)\cap [X\setminus(A\cap B)] \subset A\oplus B$.

First, we want to show that $A\oplus B\subset(A\cup B)\cap [X\setminus(A\cap B)]$. Consider an element $y \in A\oplus B$. By the definition of the symmetric sum, $y \in \{x \in A \mid x \notin B\} \cup \{x \in B \mid x \notin A\}$, therefore either $y \in \{x \in A \mid x \notin B\}$ or $y \in \{x \in B \mid x \notin A\}$. 

In both cases, $y \in A\cup B$ and $y \in [X \setminus (A \cap B)]$. Since for all $y \in A\oplus B, y \in (A\cup B)\cap [X\setminus(A\cap B)]$, this shows that $A\oplus B \subset (A\cup B)\cap [X\setminus(A\cap B)]$.

Next, we want to show that $(A\cup B)\cap [X\setminus(A\cap B)] \subset A\oplus B$. Consider an element $z \in (A\cup B)\cap [X\setminus(A\cap B)]$. Then, $z \in (A\cup B)$ and $z \in [X\setminus(A\cap B)]$. 

$z \in (A\cup B)$ implies $z \in A$ or $z \in B$. $z \in [X\setminus(A\cap B)]$ implies $z \notin A \cap B$. Therefore, either $z \in A$ and $z \notin B$ in which case $z \in A \setminus B \subset A \oplus B$, or $z \in B$ and $z \notin A$ in which case $z \in B \setminus A \subset A \oplus B$.

Thus, for all $z \in (A\cup B)\cap [X\setminus(A\cap B)]$, $z \in A\oplus B$. Hence, $(A\cup B)\cap [X\setminus(A\cap B)] \subset A\oplus B$.
 
This completes the proof.

\renewcommand\qedsymbol{QED}
\end{proof}

\item
\begin{definition}\label{powersetdef}
	Let $A$ be a set.  The ``power set'' of $A$, denoted $\wp(A),$ is the set of all subsets of $A$; that is, $\wp(A) = \{B \mid B \subset A\}$.
\end{definition}


\begin{enumerate}
	\item If $A$ is a set, show that $\wp(A) \neq \emptyset$.
 \begin{proof}
If $A$ is any set, we know that $\emptyset \subset A$. Hence, $\emptyset \in \wp(A)$, therefore $\wp(A) \not = \emptyset$.
\renewcommand\qedsymbol{QED}
 \end{proof}
	\item Let $\emptyset$ be the empty set.  Write down the elements of $\wp(\wp(\emptyset)).$

$\wp(\wp(\emptyset)) = \{\emptyset, \{\emptyset\}\}$.
\end{enumerate}






\item 


Let $A, B, C$ be subsets of $\bbN$.  Extend Theorem~\ref{basicsets} by showing that, for any $k \in \bbN$
\[\begin{split}
	&A\subset A \cup B \cup C,\\
	&A \cap B \cap C \subset A.
\end{split}
\]
Can this be extended to four sets $A,B,C,D$?  What about five?  Is there any limit?


\item  
\begin{enumerate}
\item Set
$A = \{1,2\}$, 
$B = \{3,4\}$, and
$f = \{(a,b) \mid a\in A, b \in B \}$.  \\
Write out the elements of $f$.  Is $f$ a function $A \to B$?

\item Let $C=\{(1,2),(2,2),(3,2)\}$. \\
Can $C$ be a function?  (For starters, what would $A$ and $B$ be?)

\item Write out the elements of the set $D= \{(b,a) \mid (a,b) \in C\}, $ where $C$ is as given in b). Can $D$ be a function?  
\end{enumerate}


\item Take the sets $A=\left\{1,2,3\right\}$ and $B=\left\{1,4,9\right\}$. Consider the following four statements:
\begin{enumerate}
\item For all $a\in A$, there is some $b\in B$ such that $a^2=b$.
\item There is some $b\in B$ such that, for all $a\in A$, $a^2=b$.
\item There is some $b\in B$ such that $a^2=b$ for all $a\in A$.
\item For all $a\in A$, $a^2=b$ for some $b\in B$.
\end{enumerate}
Each statement is equivalent to exactly one other in the list. Which statements are true? Which pairs are equivalent to each other?


\item
Let $f:\bbN\rightarrow\bbN$ be given by $f(n)=n^3.$
\begin{enumerate}
\item Is $f$ surjective?  Is $f$ injective?
\item Let $A\subset\bbN$ be the set $\{1,2,\ldots,30\}.$  What is $f^{-1}(f(A))?$  What is
$f(f^{-1}(A))?$
\end{enumerate}


\item 

Define $f:\bbZ\times\bbZ\rightarrow\bbZ$ by $f(m,n)=mn.$  Is $f$ injective?  Surjective?  If
$A\subset \bbZ$ is the set of even integers, what is $f^{-1}(A)?$


\item Let $f:A\to B$ and $g:B\to C$ be functions.
\begin{enumerate}
\item Suppose that $f$ and $g\circ f$ are injective. Is $g$ necessarily injective?

\begin{proof}
No. For a counterexample, consider $A = \{1\}, B = \{p,q\}, C = \{x\}$. Let $f\colon A \rightarrow B$ be defined by $f(1)=p$ and let $g\colon B \rightarrow C$ be defined by $g(p)=g(q)=x$. Then both $f$ and $g \circ f$ are injective, but $g$ is not injective.

More generally, let $B_0 \subset B$ be the image of $f$. Then, for every $b_0 \in B_0$, i.e., for every element of $B$ that is reached by $f$, there exists a unique $a \in A$ such that $f(a)=b_0$. 

Similarly, let $C_0 \subset C$ be the image of $g\circ f$. Then, for every $c_0 \in C_0$, i.e., for every element of $C$ that is reached by $g \circ f$, there exists a unique $a \in A$ such that $(g\circ f)(a)=c_0$. 

However, there could exist some $b' \in B, b' \notin B_0$ such that $g(b')=c_0$. Then $g(b_0)=g(b')=c_0$ and $b_0 \not = b'$. Hence, $g$ is not necessarily injective.

\renewcommand\qedsymbol{QED}
\end{proof}

\item Suppose that $g$ and $g\circ f$ are injective. Is $f$ necessarily injective?

\begin{proof}
Let $x \not= y$. Then, since $g \circ f$ is injective, we know that $(g \circ f)(x)\not=(g \circ f)(y)$. Further, since $g$ is injective, we know that $f(x)\not=f(y)$.

Thus, $f(x)\not=f(y)$ follows from $x \not= y$, hence we know that $f$ is necessarily injective.


\renewcommand\qedsymbol{QED}
\end{proof}

\item Suppose that $f$ and $g\circ f$ are surjective. Is $g$ necessarily surjective?

\begin{proof}



Yes. Since $f$ is surjective, $\forall b \in B, \exists a \in A$ such that $f(a)=b$.

Since $g \circ f$ is surjective, $\forall c \in C, \exists a \in A$ such that $(g \circ f)(a)=g(f(a))=c$.

Assume $g$ is not surjective. Then, there exists some $c_0 \in C$ such that for all $b \in B$, $g(b) \not = c_0$.

We know that there exists some $a_0 \in A$ such that $(g \circ f) (a_0)=c_0$, and $f(a_0)=b_0$. Hence, $g(b_0)=c_0$ and we have reached a contradiction.

\renewcommand\qedsymbol{QED}
\end{proof}

\item Suppose that $g$ and $g\circ f$ are surjective. Is $f$ necessarily surjective?

\begin{proof}
No. For a counterexample, consider $A = \{1\}, B = \{p,q\}, C = \{x\}$. Let $f\colon A \rightarrow B$ be defined by $f(1)=p$ and let $g\colon B \rightarrow C$ be defined by $g(p)=g(q)=x$. Then both $g$ and $g \circ f$ are surjective, but $f$ is not surjective.

\renewcommand\qedsymbol{QED}
\end{proof}

\end{enumerate}





\item Let $f:A\longrightarrow B$ be a function. Let $X\subset A$ and $Y\subset B.$ 
\begin{enumerate}
\item Prove that if $f$ is surjective then $f(f^{-1} (Y)))=Y.$
\begin{proof}
If $f$ is surjective, we know that for all $b \in B$, there exists some $a \in A$ such that $f(a)=b$. To denote this, for all $i \in \bbN, i \leq |B|,$ let $a_i$ correspond to $b_i$ by the relation $f(a_i)=b_i$.

Let $Y \subset B = \{b_1, b_2, \dots, b_n\}$, then we have that $f^{-1}(Y)=\{a_1, a_2, \dots, a_n\}$. Hence,  $f(f^{-1}(Y))=\{b_1, b_2, \dots, b_n\} = Y$.

\renewcommand\qedsymbol{QED}
\end{proof}

 \item Prove that if $f$ is injective then $f^{-1}(f(X))=X.$
 \begin{proof}
 If $f$ is injective, we know that $f(a)=f(a')$ implies $a=a'$. 
 
Let $X \subset A = \{a_1, a_2, \dots, a_n\}$. Since $f$ is one-to-one, we know that the number of elements in $f(X)$ must be the same as the number of elements in $X$, i.e., $|f(X)|=|X|$. Hence, let $f(X)=\{b_1, b_2, \dots, b_n\}$. 

Assume $b_i, b_k \in f(X)$ and $b_i = b_k$, i.e., $f(a_i)=f(a_k)$. Then, since $f$ is injective, $a_i =a_k$.

Since $f$ is a function, any particular $a_i \in X$ cannot map to two or more different $b_i \in f(X)$. Since $f$ is injective, for every $b_i \in f(X), \exists! a_i\in X$ such that $f(a_i)=b_i$. Hence, we know that $f^{-1}(f(X))=\{a_1, a_2, \dots, a_n\}=X$.

\renewcommand\qedsymbol{QED}
\end{proof}
 \item Are the converse statements also true? i.e. If $f(f^{-1} (Y)))=Y$ for all subsets $Y\subset B,$ must $f$ be surjective? If $f^{-1}(f(X))=X$ for all subsets $X\subset A,$ must $f$ be injective?
 \begin{proof}

Yes. Both the converse statements are true. We prove them by contraposition.

In the first case, we assume that $f$ is not surjective, and then show that there exists some $Y \subset B$ such that $f(f^{-1} (Y)))\not =Y$. We present a demonstration by counterexample.

Let $A = \{1\}, B=\{p,q\}, Y=\{p,q\}$, and let $f\colon A \rightarrow B$ be defined by $f(1)=p$. Since $\neg \exists a\in A$ such that $f(a)=q$, $f$ is not surjective. Then, $f^{-1}(Y)=\{1\}$ and $f(f^{-1}(Y))=f(1)=p \not = Y$.

In the second case, we assume that $f$ is not injective, and then show that there exists some $X \subset A$ such that $f^{-1}(f(X))\not =X$. We present a demonstration by counterexample.

Let $A = \{1,2\}, B=\{p\}, X=\{1\}$, and let $f\colon A \rightarrow B$ be defined by $f(1)=p, f(2)=p$. Then $f(X)=f(1)=p$ and $f^{-1}(f(X))=f^{-1}(p)=\{1, 2\} \not= X$.

\renewcommand\qedsymbol{QED}
\end{proof}
 \end{enumerate}


\item Let $f:A\longrightarrow B$ be a bijection. Let $g$ be the inverse function to $f,$ given by Proposition 1.27.  Let $Y\subset B.$ Show that 
$g(Y)=f^{-1}(Y).$ 

Note: $g(Y)$ denotes the image of $Y$ under the map $g$ and $f^{-1}(Y)$ denotes the preimage of $Y$ under $f.$ Thus when $g=f^{-1}$ exists as a function, the two possible interpretations of $f^{-1}(Y)$ coincide.

\begin{proof}
Define $g\colon B \rightarrow A$ as $g(b)=a_b$ for all $b \in B$, where $a_b$ is the unique element of $A$ such that $f(a_b)=b$. 

Let $Y \subset B = \{b_1, b_2, \dots, b_n\}$, then we have that the preimage $f^{-1}(Y)=\{a_{b1}, a_{b2}, \dots, a_{bn}\}$. 

By the definition of $g$, the image of $Y$ under the map $g$ is $g(Y)=a_b$ for all $b \in Y$. Hence, $g(Y)=\{a_{b1}, a_{b2}, \dots, a_{bn}\}$.

This completes the proof.

\renewcommand\qedsymbol{QED}
\end{proof}








\item Recall the definition of power set, Definition~\ref{powersetdef}.
\begin{enumerate}
\item Let $A$ be any set.  Show that there is no bijection between $A$ and its power set $\wp(A).$
(Hint:  If $f:A\rightarrow \wp(A)$ is any function, think about the set
$B=\{a\in A\mid a\not\in f(a)\}\subset A.)$
\begin{proof}
We prove this by contradiction. Assume that there exists a bijection $f:A\rightarrow \wp(A)$. We show that $f$ cannot be surjective.

Let $B \subset A$ be defined by $B=\{a\in A\mid a\not\in f(a)\}$. Then, $B \in \wp(A)$. 

If $f$ is surjective, then for all $Y \in \wp(A)$, there exists some $a \in A$ such that $f(a)=Y$. We want to show that there exists some $Y_0 \in \wp(A)$ such that for all $a \in A, f(a) \not = Y_0$.

Let $Y_0 = B$. Assume that there exists some $a_0 \in A$ such that $f(a_0) = B$. 

If $a_0 \in B$, then by definition $a_0 \notin f(a_0)$, hence $f(a_0) \not = B$. If $a_0 \notin B$, then by definition $a_0 \in f(a_0)$, hence $f(a_0) \not = B$.

Therefore, $f$ is not surjective, hence it is not bijective. This completes the proof.


\renewcommand\qedsymbol{QED}
\end{proof}

The above is my official submission for the homework assignment. Out of curiosity, I have attempted another proof, which I am not sure is fully valid/rigorous. 

\begin{proof}[Second Proof Attempt]
Assume that there exists a bijection $f \colon A \rightarrow \wp(A)$. Then, there must exist a bijection $g \colon \wp(A) \rightarrow A$, i.e., $g$ must be both injective and surjective. We show that $g$ cannot be injective using the fact that for all $j,k \in \bbN$, there does not exist an injection from $[j+k]$ to $[j]$.

Let $|A|=k$, then we know that there exists a bijection $h \colon A \rightarrow [k]$. 

By definition, $\wp(A)$ is the set of all subsets of $A$. Hence, for all $a \in A$, and for all $A' \in \wp(A)$, either $a \in A'$ or $a \notin A'$. Therefore, $|\wp(A)| = 2^{|A|}$. 

In Script 0, we showed that $(1+x)^n \geq 1 + nx$ for all $x > -1, n\in \bbN$. Substituting $x = 1$, we get that $2^n \geq 1+n$ for all $n \in \bbN$. Hence, $2^{|A|} \geq 1 + |A|$. Therefore, $|\wp(A)| \geq 1+ |A|$, i.e, there exists some bijection $q \colon \wp(A) \rightarrow [1+k]$. Hence, there also exists some bijection $q' \colon [1+k] \rightarrow \wp(A)$.

Therefore, if $g \colon \wp(A) \rightarrow A$ is bijective, then $h\circ g \colon \wp(A) \rightarrow [k]$ must be bijective. 

Hence, $h \circ g \circ q' \colon [1+k] \rightarrow [k]$ must be bijective. We know that $h \circ g \circ q'$ cannot be injective, hence it cannot be bijective.

We have reached a contradiction, and this completes the proof.



\renewcommand\qedsymbol{QED}
\end{proof}
 
\item
Show that there is no injective map from $\wp(\bbN)$ to $\bbN.$

\begin{proof}
Let $g \colon \wp(A) \rightarrow A$ be a function. We show that $g$ cannot be injective using the fact that for all $j,k \in \bbN$, there does not exist an injection from $[j+k]$ to $[j]$.

Let $|A|=k$, then we know that there exists a bijection $h \colon A \rightarrow [k]$. 

By definition, $\wp(A)$ is the set of all subsets of $A$. Hence, for all $a \in A$, and for all $A' \in \wp(A)$, either $a \in A'$ or $a \notin A'$. Therefore, $|\wp(A)| = 2^{|A|}$. 

In Script 0, we showed that $(1+x)^n \geq 1 + nx$ for all $x > -1, n\in \bbN$. Substituting $x = 1$, we get that $2^n \geq 1+n$ for all $n \in \bbN$. Hence, $2^{|A|} \geq 1 + |A|$. Therefore, $|\wp(A)| \geq 1+ |A|$, i.e, there exists some bijection $q \colon \wp(A) \rightarrow [1+k]$. Hence, there also exists some bijection $q' \colon [1+k] \rightarrow \wp(A)$.

Therefore, if $g \colon \wp(A) \rightarrow A$ is injective, then $h\circ g \colon \wp(A) \rightarrow [k]$ must be injective (since $h$ is bijective). 

Hence, $h \circ g \circ q' \colon [1+k] \rightarrow [k]$ must be injective (since $q'$ is bijective). We have reached a contradiction, and this completes the proof.

\renewcommand\qedsymbol{QED}
\end{proof}
\end{enumerate}

\item Let $A$ be a set with cardinality $n.$ Let $f\colon [n]\longrightarrow A$ be a bijection. Show that $A=\{f(1),f(2),\cdots,f(n)\}$ and deduce that we can write $A=\{a_1,a_2,\cdots, a_n\}.$ 

\item Let $f: A \to B$ be a function. 
\begin{enumerate}
\item Let $X$ and $Y$ be subsets of $A$. Is it true that $f(X \cup Y) = f(X) \cup f(Y)$? Is it true that  
$f(X\cap Y)= f(X)\cap f(Y)?$ Either prove or give a counterexample in each case.
\begin{proof}[Proof for Unions]
Suppose $m \in f(X \cup Y)$. Then, $m = f(k)$ for some $k \in X \cup Y$. If $k \in X$, then $m \in f(X)$. If $k \in Y$, then $m \in f(Y)$. Therefore, $m \in f(X) \cup f(Y)$. Hence, $f(X \cup Y) \subset [f(X) \cup f(Y)]$.

Suppose $m \in f(X) \cup f(Y)$. If $m \in f(X)$, then $m=f(k)$ for some $k \in X$, i.e., $m=f(k)$ for some $k \in X \cup Y$, hence $m \in f(X \cup Y)$. If $m \in f(Y)$, then $m=f(k)$ for some $k \in Y$, i.e., $m=f(k)$ for some $k \in X \cup Y$, hence $m \in f(X \cup Y)$. Hence, $[f(X) \cup f(Y)] \subset f(X \cup Y)$.

Therefore, $f(X \cup Y) = f(X) \cup f(Y)$.


\renewcommand\qedsymbol{QED}
\end{proof}

\begin{proof}[Proof for Intersections]
We present a counterexample. Let $A = \{1,2\}, B=\{p\}$. Let $X \subset A, Y \subset A$ be defined by $X = \{1\}, Y=\{2\}$. 

Then, $X \cap Y = \emptyset$, therefore $f(X \cap Y) = \emptyset$. 

$f(X)=f(Y)=\{p\}$, hence $f(X)\cap f(Y)=\{p\}$. 

$\{p\} \not = \emptyset$. This completes the proof by counterexample.

\renewcommand\qedsymbol{QED}
\end{proof}

\item  Now let $X$ and $Y$ be subsets of $B$. Is it true that $f^{-1}(X \cup Y) = f^{-1}(X) \cup f^{-1}(Y)$? Is it true that $f^{-1}(X\cap Y)=f^{-1}(X)\cap f^{-1} (Y)?$ Either prove or give a counterexample in each case.

\begin{proof}[Proof for Unions]
Suppose $m \in f^{-1}(X \cup Y)$. Then, $f(m) \in X \cup Y$. If $f(m) \in X$, then $m \in f^{-1}(X)$, hence $m \in f^{-1}(X) \cup f^{-1}(Y)$. If $f(m) \in Y$, then $m \in f^{-1}(Y)$, hence $m \in f^{-1}(X) \cup f^{-1}(Y)$. Hence, $f^{-1}(X \cup Y) \subset [f^{-1}(X) \cup f^{-1}(Y)]$.

Suppose $m \in f^{-1}(X) \cup f^{-1}(Y)$. If $m \in f^{-1}(X)$, then $f(m) \in X$, hence $f(m) \in X \cup Y$, therefore $m \in f^{-1}(X \cup Y)$. If $m \in f^{-1}(Y)$, then $f(m) \in Y$, hence $f(m) \in X \cup Y$, therefore $m \in f^{-1}(X \cup Y)$. Hence, $[f^{-1}(X) \cup f^{-1}(Y)] \subset f^{-1}(X \cup Y)$.

Therefore, $f^{-1}(X \cup Y) = f^{-1}(X) \cup f^{-1}(Y)$.

\renewcommand\qedsymbol{QED}
\end{proof}

\begin{proof}[Proof for Intersections]
Suppose $m \in f^{-1}(X \cap Y)$. Then, $f(m) \in X \cap Y$, i.e., $f(m) \in X$ and $f(m) \in Y$. Hence, $m \in f^{-1}(X)$ and $m \in f^{-1}(Y)$, therefore $m \in f^{-1}(X) \cap f^{-1}(Y)$. Hence, $f^{-1}(X \cap Y) \subset [f^{-1}(X) \cap f^{-1}(Y)]$.

Suppose $m \in f^{-1}(X) \cap f^{-1}(Y)$. Then, $m \in f^{-1}(X)$ and $m \in f^{-1}(Y)$. Hence, $f(m) \in X$ and $f(m) \in Y$, so $f(m) \in X \cap Y$. Therefore, $m \in f^{-1}(X \cap Y)$. Hence, $[f^{-1}(X) \cap f^{-1}(Y)] \subset f^{-1}(X \cap Y)$.

Therefore, $f^{-1}(X\cap Y)=f^{-1}(X)\cap f^{-1} (Y)$.


\renewcommand\qedsymbol{QED}
\end{proof}
\end{enumerate}



\item
Suppose that $A,B\subset \mathbb{N}.$ Prove that if $A$ and $B$ are finite and there is a bijection $f:A\to B,$ then $|A|=|B|.$


\item
	Prove that there is no set $A$ with maximal cardinality.  In other words, show that there does not exist a set $A$ with the property that if $B$ is a set, then $B$ has smaller cardinality than $A$.





\item

\begin{enumerate}
\item[a)]  
Let $X$ be a countable set and $Y$ a finite set such that $X\cap Y=\emptyset$. Show that $X\cup Y$ is countable. 
\begin{proof}
Case 1: $X$ is finite. Then, let $a = |X|$ and $b = |Y|$, then we know that $|X \cup Y| = a + b$. Let $c \in \bbN, c = a+b$, then we know that $X \cup Y$ is countable since there exists a bijection from $X \cup Y$ to $[c]$.

Case 2: $X$ is infinite. We know that there exists a bijection $f \colon X \rightarrow \bbN$ and there exists a bijection from $Y$ to $[b]$, where $b = |Y|$. Then, define $g \colon X \cup Y \rightarrow \bbN$ by 
\[
g(x) =
\begin{cases}
i, \text{for } x_i \in Y \\
f(x_j)+b, \text{for } x_j \notin Y.
\end{cases}
\]
for all $x \in X \cup Y$. (Note: we know that $X \cap Y = \emptyset$, hence $\forall i, \forall j, x_i \not = x_j$.)

We first show that $g$ is surjective. For all $n \in \bbN$, there exists an $x_n \in X \cup Y$ such that $g(x_n)=n$. This is given by
\[
x_n =
\begin{cases}
x_{i=n} \text{ if } n \leq b\\
f^{-1}(n - b), \text{if } n > b.
\end{cases}
\]

Next, we show that $g$ is injective.

Case 1: $x_i,x_{i'} \in Y, x_i \not = x_{i'}$. Then, $g(x_i)=i \not = i' = g(x_{i'})$.

Case 2: $x_j,x_{j'} \notin Y, x_j \not = x_{j'}$. Then, since $f$ is injective, $f(x_j) \not= f(x_{j'})$, hence $f(x_j)+b \not= f(x_{j'})+b$, therefore $g(x_j) \not= g(x_{j'})$.

Case 3: $x_i \in Y, x_j \notin Y$ (it follows that $x_i \not = x_j$). Then, $g(x_i)=i \leq b$ while $g(x_j)=f(x_j)+b > b$, therefore $g(x_i) \not= g(x_{j})$.

This completes the proof.

\renewcommand\qedsymbol{QED}
\end{proof}


 \item[b)] Prove that a union of 2 disjoint countable sets is countable.
\begin{proof}
Let $X \cap Y =\emptyset$ where $X$ and $Y$ are countable. Then, we want to show that $X \cup Y$ is countable.

Case 1: if $X,Y$ are finite, then $|X \cup Y| = |X|+|Y|$, hence $X \cup Y$ is countable.

Case 2: if $X$ is finite and $Y$ is countable or vice versa, then by part a), $X \cup Y$ is countable.

Case 3: if $X,Y$ are infinite and countable, then we know that there exists a bijection $f \colon X \rightarrow \bbN$ and there exists a bijection $g \colon Y \rightarrow \bbN$.

Then, we define $h \colon X\cup Y \rightarrow \bbN$ as
\[
h(m) =
\begin{cases}
(2 \cdot f(m))-1 \text{ if } m \in X\\
2 \cdot g(m), \text{if } m \in Y.
\end{cases}
\]

First, we show that $h$ is surjective. For all $n \in \bbN$,
\[
n =
\begin{cases}
h(m_i) \text{ such that } m \in Y, i = \frac{n}{2} \text{ if } n \text{ is even}\\
h(m_j) \text{ such that } m \in X, j = \frac{n+1}{2} \text{ if } n \text{ is odd}.
\end{cases}
\]

Next, we show that $h$ is injective. 

Case 1: $m \in Y, m' \in X$, then $h(m)$ is even and $h(m')$ is odd, hence $h(m)\not = h(m')$.

Case 2: $m \in Y, m' \in Y, m\not = m'$. Then, $h(m)=2\cdot g(m) \not = 2 \cdot g(m') = h(m')$ since $g$ is injective.

Case 3: $m \in X, m' \in X, m\not = m'$. Then, $h(m)=(2\cdot f(m))-1 \not = (2 \cdot f(m'))-1 = h(m')$ since $f$ is injective.

This completes the proof.

\renewcommand\qedsymbol{QED}
\end{proof}
  \item[c)] Use a) and b) to prove that $\bbZ$ is countable.

\begin{proof}
Let $\bbN_0 = \{-1,-2,-3, \dots \}$ be the set of negative integers. Then, we know that $\bbN_0$ is countable, since there exists a bijection $f \colon \bbN_0 \rightarrow \bbN$ defined as $f(n)= -n$ for all $n \in \bbN_0$.

We also know that $\bbZ = \bbN \cup \bbN_0 \cup \{0\}$. 

By part a), $\bbN_0$ is countable and $\{0\}$ is finite, and they are disjoint, hence $\bbN_0 \cup \{0\}$ is countable.

By part b), $\bbN$ is countable and $\bbN_0 \cup \{0\}$ is countable, and they are disjoint, hence $\bbN \cup \bbN_0 \cup \{0\}$ is countable. Therefore, $\bbZ$ is countable.

\renewcommand\qedsymbol{QED}
\end{proof}

 \item[d)] Suppose that $A_1, A_2, \dots$ are sets such that $A_i$ is countable for each $i \in\bbN$. Show that, for each $n$, \[\bigcup_{i=1}^n A_i\] is countable.

\begin{proof}
Case 1: $n=1$, i.e., $\bigcup_{i=1}^n A_i = \bigcup_{i=1}^1 A_i = A_1$. Then, by definition, $A_1$ is countable. 

Case 2: $n>1$. First, we rewrite this to construct a finite union of disjoint countable sets, so that we can use the result from part b) by induction on $n$. 

Let $C = \bigcap_{i=1}^n A_i$, then we know that $C$ is countable. Then, $\bigcup_{i=1}^n A_i = \bigcup_{i=1}^n (A_i \setminus C) \cup C$. 

By part b), we know that if $\bigcup_{i=1}^n (A_i \setminus C)$ is countable, then so is $\bigcup_{i=1}^n (A_i \setminus C) \cup C$. 

For all $i \geq 1, i \leq n$, let $A_i \setminus C$ be denoted by $M_i$. Hence, we want to show that $\bigcup_{i=1}^n M_i$ is countable, where each $M_i$ is countable and $\bigcap_{i=1}^n M_i = \emptyset$.

We prove this by induction on $n$. By part b), the base case is true for $n = 2$. 

Next, we want to show that if $\bigcup_{i=1}^n M_i$ is countable, then $\bigcup_{i=1}^{n+1} M_i$ is countable.

We know that $\bigcup_{i=1}^{n+1} M_i = \bigcup_{i=1}^n M_i \cup M_{n+1}$. By the inductive hypothesis, $\bigcup_{i=1}^n M_i$ is countable and $M_{n+1}$ is countable, hence by part b), $\bigcup_{i=1}^{n+1} M_i$ is countable.

This completes the proof.

\renewcommand\qedsymbol{QED}
\end{proof}

     \item[e)] Prove that a countable union of countable sets is countable. That is, if $\{A_i\}_{i\in\bbN} $ is a countable  collection of countable sets, then 
  $$\bigcup_{i\in\bbN} A_i$$ 
  is countable.

\begin{proof}
Let the countable sets $A_i$ be indexed by $i$ and let the elements of each of the countable sets $A_i$ be indexed by $j$. Then, every element in the countable union of countable sets (i.e., every element in $\bigcup_{i\in\bbN} A_i$) can be specified as $a_{ij}$ (i.e., the $j$th element of the $i$th set). For example, $A_1 = \{a_{11}, a_{12}, a_{13}, \dots \}$.

Let $M = \bigcup_{i\in\bbN} A_i$, then it suffices to show that there exists an injection $f \colon M \rightarrow \bbN \times \bbN$, since we know that $\bbN \times \bbN$ is countable.

For all $m \in M$, let $f(m)=(i,j)$. We want to show that $f$ is injective.

Case 1: $m, m' \in A_i, m \not= m'$. It follows that $m = a_{ij}$ and $m' = a_{ij'}$ for some $j,j'$. Then, $f(m) = (i,j) \not= (i,j') = f(m')$.

Case 2: $m \in A_i, m' \in A_{i'}$. It follows that $m \not= m'$. We know that $m = a_{ij}$ and $m' = a_{i'j'}$.

Subcase 2a: $j = j'$. Then, $f(m) = (i,j) \not= (i',j) = f(m').$

Subcase 2b: $j \not= j'$. Then, $f(m) = (i,j) \not= (i',j') = f(m').$

Therefore, $f$ is injective and this completes the proof.

\renewcommand\qedsymbol{QED}
\end{proof}
  
  \item[f)]  Prove that if $A$ and $B$ are countable then $A\times B$ is countable.

\begin{proof}
We know that there exists a bijection $f \colon A \rightarrow \bbN$ and there exists a bijection $g \colon B \rightarrow \bbN$. 

Then, define $h \colon A \times B \rightarrow \bbN \times \bbN$ by $h(a,b)=(f(a),g(b))$. We know that $\bbN \times \bbN$ is countable, hence it suffices to show that $h$ is injective.

Suppose $a \not = a'$, then $h(a,b)=(f(a),g(b)) \not= (f(a'),g(b)) = h(a',b)$ since $f$ is injective. Analogously, suppose $b \not = b'$, then $h(a,b)=(f(a),g(b)) \not= (f(a),g(b')) = h(a,b')$ since $g$ is injective. 

Similarly, if $a \not = a'$ and $b \not = b'$, then $f(a) \not= f(a')$ and $g(b) \not= g(b')$, therefore $h(a,b) =(f(a),g(b)) \not =(f(a'),g(b')) = h(a',b')$.

Therefore, $h$ is injective and this completes the proof.

\renewcommand\qedsymbol{QED}
\end{proof}

  \item[g)] Prove that if $A_1,A_2,\cdots, A_n$ are countable, then so is $A_1\times A_2\times \cdots\times A_n.$

\begin{proof}
We prove this by induction on $n$.

For the base case, we want to show that if $A_1$ and $A_2$ are countable, then so is $A_1 \times A_2$. This is established in part f).

For the inductive step, we want to show that if $A_1 \times A_2 \times \dots \times A_n$ is countable and $A_{n+1}$ is countable, then $A_1 \times A_2 \times \dots \times A_n \times A_{n+1}$ is countable.

Let $M = A_1 \times A_2 \times \dots \times A_n$, where $M$ is countable. By part f), if $A_{n+1}$ is countable, then $M \times A_{n+1}$ is countable.

This completes the proof.

\renewcommand\qedsymbol{QED}
\end{proof}

  \item[h)] Let $A_n=\{0,1\},$ for every $n.$ Show that $A_1\times A_2\times\cdots\times A_n\times\cdots$ is uncountable.

\begin{proof}
We know that $\wp(\bbN)$ is uncountable, hence it suffices to show that there exists a bijection $f \colon A_1\times A_2\times\cdots\times A_n\times\cdots \rightarrow \wp(\bbN)$. We prove this by induction on $n$.

Define $f_n \colon  A_1\times A_2\times\cdots\times A_n \rightarrow \wp([n])$ as $f_n(a_i) = \{i \in [n] \mid a_i = 1 \}$.

(Side note: The intuition behind this formalism is that every element of $\wp(\bbN)$ is mapped by the string of the ordered pair such that if the $i$th element of the ordered pair is 0, that element is not in $L \subset \bbN, L \in \wp(\bbN)$; and if the $i$th element of the ordered pair is 1, that element is in $L \subset \bbN, L \in \wp(\bbN)$. Here, I use the term ordered pair even though it has more than 2 elements and so is not strictly speaking a pair).

For the base case, $f_2\colon A_1 \times A_2 \rightarrow \wp([2])$. Then, $f_2(0,0) = \emptyset, f_2(0,1) = \{2\}, f_2(1,0)=\{1\}, f_2(1,1) = \{1,2\}$. Hence, $f$ is bijective.

For the inductive step, we want to show that if $f_n \colon A_1\times A_2\times\cdots\times A_n \rightarrow \wp([n])$ is bijective, then $f_{n+1} \colon A_1\times A_2\times\cdots\times A_{n+1} \rightarrow \wp([n+1])$ is bijective.

We note that $f_{n+1}(a_i) = \{i \in [n+1] \mid a_i = 1 \}$.

Case 1: $a_{n+1} = 0$. Then, $f_{n+1} (a) = f_n(a)$ for all $a \in A_1\times A_2\times\cdots\times A_{n+1}$. Hence, if $f_n$ is bijective then $f_{n+1}$ is bijective.

Case 2: $a_{n+1} = 1$. Then, $f_{n+1} (a) = \{f_n(a) \cup a_{n+1}\}$ for all $a \in A_1\times A_2\times\cdots\times A_{n+1}$. Hence, if $f_n$ is bijective then $f_{n+1}$ is bijective.

This completes the proof.

\renewcommand\qedsymbol{QED}
\end{proof}

 \end{enumerate}





\end{enumerate}



\end{document}
