\documentclass[11pt]{article}

\usepackage{color}
%\input{rgb}
%----------Packages----------
\usepackage{amsmath}
\usepackage{amssymb}
\usepackage{amsthm}
\usepackage{amsrefs}
\usepackage{dsfont}
\usepackage{mathrsfs}
\usepackage{stmaryrd}
\usepackage[mathcal]{eucal}
\usepackage[all]{xy}
\usepackage{verbatim}  %%includes comment environment
\usepackage{fullpage}  %%smaller margins
%----------Commands----------

%%penalizes orphans
\clubpenalty=9999
\widowpenalty=9999





%% bold math capitals
\newcommand{\bA}{\mathbf{A}}
\newcommand{\bB}{\mathbf{B}}
\newcommand{\bC}{\mathbf{C}}
\newcommand{\bD}{\mathbf{D}}
\newcommand{\bE}{\mathbf{E}}
\newcommand{\bF}{\mathbf{F}}
\newcommand{\bG}{\mathbf{G}}
\newcommand{\bH}{\mathbf{H}}
\newcommand{\bI}{\mathbf{I}}
\newcommand{\bJ}{\mathbf{J}}
\newcommand{\bK}{\mathbf{K}}
\newcommand{\bL}{\mathbf{L}}
\newcommand{\bM}{\mathbf{M}}
\newcommand{\bN}{\mathbf{N}}
\newcommand{\bO}{\mathbf{O}}
\newcommand{\bP}{\mathbf{P}}
\newcommand{\bQ}{\mathbf{Q}}
\newcommand{\bR}{\mathbf{R}}
\newcommand{\bS}{\mathbf{S}}
\newcommand{\bT}{\mathbf{T}}
\newcommand{\bU}{\mathbf{U}}
\newcommand{\bV}{\mathbf{V}}
\newcommand{\bW}{\mathbf{W}}
\newcommand{\bX}{\mathbf{X}}
\newcommand{\bY}{\mathbf{Y}}
\newcommand{\bZ}{\mathbf{Z}}

%% blackboard bold math capitals
\newcommand{\bbA}{\mathbb{A}}
\newcommand{\bbB}{\mathbb{B}}
\newcommand{\bbC}{\mathbb{C}}
\newcommand{\bbD}{\mathbb{D}}
\newcommand{\bbE}{\mathbb{E}}
\newcommand{\bbF}{\mathbb{F}}
\newcommand{\bbG}{\mathbb{G}}
\newcommand{\bbH}{\mathbb{H}}
\newcommand{\bbI}{\mathbb{I}}
\newcommand{\bbJ}{\mathbb{J}}
\newcommand{\bbK}{\mathbb{K}}
\newcommand{\bbL}{\mathbb{L}}
\newcommand{\bbM}{\mathbb{M}}
\newcommand{\bbN}{\mathbb{N}}
\newcommand{\bbO}{\mathbb{O}}
\newcommand{\bbP}{\mathbb{P}}
\newcommand{\bbQ}{\mathbb{Q}}
\newcommand{\bbR}{\mathbb{R}}
\newcommand{\bbS}{\mathbb{S}}
\newcommand{\bbT}{\mathbb{T}}
\newcommand{\bbU}{\mathbb{U}}
\newcommand{\bbV}{\mathbb{V}}
\newcommand{\bbW}{\mathbb{W}}
\newcommand{\bbX}{\mathbb{X}}
\newcommand{\bbY}{\mathbb{Y}}
\newcommand{\bbZ}{\mathbb{Z}}

%% script math capitals
\newcommand{\sA}{\mathscr{A}}
\newcommand{\sB}{\mathscr{B}}
\newcommand{\sC}{\mathscr{C}}
\newcommand{\sD}{\mathscr{D}}
\newcommand{\sE}{\mathscr{E}}
\newcommand{\sF}{\mathscr{F}}
\newcommand{\sG}{\mathscr{G}}
\newcommand{\sH}{\mathscr{H}}
\newcommand{\sI}{\mathscr{I}}
\newcommand{\sJ}{\mathscr{J}}
\newcommand{\sK}{\mathscr{K}}
\newcommand{\sL}{\mathscr{L}}
\newcommand{\sM}{\mathscr{M}}
\newcommand{\sN}{\mathscr{N}}
\newcommand{\sO}{\mathscr{O}}
\newcommand{\sP}{\mathscr{P}}
\newcommand{\sQ}{\mathscr{Q}}
\newcommand{\sR}{\mathscr{R}}
\newcommand{\sS}{\mathscr{S}}
\newcommand{\sT}{\mathscr{T}}
\newcommand{\sU}{\mathscr{U}}
\newcommand{\sV}{\mathscr{V}}
\newcommand{\sW}{\mathscr{W}}
\newcommand{\sX}{\mathscr{X}}
\newcommand{\sY}{\mathscr{Y}}
\newcommand{\sZ}{\mathscr{Z}}


\renewcommand{\phi}{\varphi}

\renewcommand{\emptyset}{\O}

\providecommand{\abs}[1]{\lvert #1 \rvert}
\providecommand{\norm}[1]{\lVert #1 \rVert}


\providecommand{\ar}{\rightarrow}
\providecommand{\arr}{\longrightarrow}

\renewcommand{\_}[1]{\underline{ #1 }}


\DeclareMathOperator{\ext}{ext}

\newcommand{\head}[1]{
	\begin{center}
		{\large #1}
		\vspace{.2 in}
	\end{center}
	
	\bigskip 
}
\newcommand{\hint}[2][Hint]{
	
	(#1: #2)
}

%----------Theorems----------

\newtheorem{theorem}{Theorem}[section]
\newtheorem{proposition}[theorem]{Proposition}
\newtheorem{lemma}[theorem]{Lemma}
\newtheorem{corollary}[theorem]{Corollary}


\newtheorem*{axiom4}{Axiom 4}


\theoremstyle{definition}
\newtheorem{definition}[theorem]{Definition}
\newtheorem{nondefinition}[theorem]{Non-Definition}
\newtheorem{exercise}[theorem]{Exercise}
\newtheorem{remark}[theorem]{Remark}
\newtheorem{warning}[theorem]{Warning}

\newtheorem{exercise*}{Exercise 3.26}

\numberwithin{equation}{subsection}


%----------Title-------------

\newcommand{\hide}[1]{{\color{red} #1}} 
\newcommand{\com}[1]{{\color{blue} #1}} 
\newcommand{\meta}[1]{{\color{green} #1}} 
\begin{document}

\head
{MATH 161, Autumn 2024\\ SCRIPT 5: Connectedness and Boundedness} 





\setcounter{section}{5}   



We introduce a new axiom for a continuum $C$ and derive many interesting properties from it.  From now on, we will always assume that $C$ satisfies axiom 4 (as well as axioms 1,2 and 3).

\medskip

\begin{axiom4}
A continuum is connected.
\end{axiom4}



\begin{theorem}
The only subsets of a continuum $C$ that are both open and closed are $\emptyset$ and~$C$.
\end{theorem}
\begin{proof}
Assume $M \subset C$ such that $M \neq C, M \neq \emptyset$, and $M$ is both open and closed. Then, $C \setminus M$ must be both open and closed.

Since we know that $M \subset C$, there must be some $x \in C$ such that $x \notin M$, i.e., $x \in C\setminus M$.

Then, $C = M \cup (C \setminus M)$, where $M$ and $C \setminus M$ are nonempty, disjoint, and open. Hence, $C$ is disconnected, and this contradicts Axiom 4.

\renewcommand\qedsymbol{QED}
\end{proof}

\begin{corollary}  Every region is infinite.
\begin{proof}
Let $R = \_{ab}$ be a finite region. We know that all regions are open, hence $R$ is open. Finite sets are closed, hence $R$ is closed.

$a \in C, a \notin R$, hence $R \neq C$. We know that all regions are nonempty, hence $R \neq \emptyset$.

Then, $R$ is a set that is nonempty, not equal to $C$, and is both open and closed. This is a contradiction.

\renewcommand\qedsymbol{QED}
\end{proof}
\end{corollary}

\begin{corollary}  Every point of $C$ is a limit point of $C$.  

\begin{proof}
Let $x \in C$, then we want to show that $x \in LP(C)$. We know that every region is infinite, hence any region that contains $x$ must contain other points.

Hence, for all regions $R$ such that $x \in R$, we have that $R \cap (C \setminus \{x\}) \neq \emptyset.$ Hence, $x \in LP(C)$. This completes the proof.

\renewcommand\qedsymbol{QED}
\end{proof}

\end{corollary}

\begin{corollary}  Every point of the region $\_{ab}$ is a limit point of $\_{ab}$.
\begin{proof}
Suppose there exists some $x \in \_{ab}$ such that $x \notin LP(\_{ab})$. Then, there exists a region $R \in C$ such that $x \in R$ and $R \cap (\_{ab} \setminus \{x\}) = \emptyset.$

Then, $S = R \cap \_{ab}$ is a region such that $S = \{x\}$. 

Hence, $S$ is finite, but all regions are infinite. This is a contradiction.

\renewcommand\qedsymbol{QED}
\end{proof}
\end{corollary}




We will now introduce boundedness.   The first definition should be intuitively clear.  The second is subtle and powerful.  


\begin{definition}  Let $X$ be a subset of $C$.  A point $u$ is called an \emph{upper bound} of $X$ if for all $x \in X$, $x \leq u$.  A point $l$ is called a \emph{lower bound} of $X$ if for all $x \in X$, $l \leq x$.  If there exists an upper bound of $X$, then we say that $X$ is \emph{bounded above}.  If there exists a lower bound of $X$, then we say that $X$ is \emph{bounded below}.  If $X$ is bounded above and below, then we simply say that $X$ is \emph{bounded}.
\end{definition}


\begin{definition}  Let $X$ be a subset of $C$.  We say that $u$ is a \emph{least upper bound} of $X$ and write $u = \sup X$ if:
\begin{enumerate}
\item  $u$ is an upper bound of $X$, and
\item  if $u'$ is an upper bound of $X$, then $u \leq u'$.
\end{enumerate}
We say that $l$ is a \emph{greatest lower bound} and write $l = \inf X$ if:
\begin{enumerate}
\item $l$ is a lower bound of $X$, and
\item if $l'$ is a lower bound of $X$, then $l' \leq l$.
\end{enumerate}
\end{definition}

\noindent The notation $\sup$ comes from the word \emph{supremum}, which is another name for least upper bound.  The notation $\inf$ comes from the word \emph{infimum}, which is another name for greatest lower bound.

\begin{exercise}  If $\sup X$ exists, then it is unique, and similarly for $\inf X$.
\begin{proof}
We want to show that if $u = \sup X, u' = \sup X$, then $u = u'$.

Since $u$ is an upper bound and $u'$ is a supremum, we have that $u \leq u'$.

Since $u'$ is an upper bound and $u$ is a supremum, we have that $u' \leq u$.

Hence, $u = u'$. Analogous reasoning for infimum completes the proof.

\renewcommand\qedsymbol{QED}
\end{proof}
\end{exercise}

\begin{exercise} If $X$ has a first point $L,$ then $\inf X$ exists and equals $L.$ Similarly, if $X$ has a last point $U,$ then $\sup X$ exists and equals $U.$
\begin{proof}
Since $L$ is a first point, for all $x \in X, L \leq x$. Thus, $L$ is a lower bound. Hence, if $L'$ is a lower bound, $L' \leq L$. Hence, $\inf X = L$.

Analogous reasoning for infimum completes the proof.

\renewcommand\qedsymbol{QED}
\end{proof}
\end{exercise}


\begin{exercise} For this exercise, we assume that $C=\bbR.$ Find $\sup X$ and $\inf X$ for each of the following subsets
of $\bbR,$ or state that they do not exist. You need not give proofs.
\begin{enumerate}
\item $X=\bbN$
\begin{proof}
$\sup$ does not exist. $\inf = 1$.

\renewcommand\qedsymbol{QED}
\end{proof}
\item $X=\bbQ$
\begin{proof}
$\sup$ and $\inf$ do not exist.

\renewcommand\qedsymbol{QED}
\end{proof}
\item $X=\{\frac1n\mid n\in\bbN\}$
\begin{proof}
$\sup = 1$ and $\inf = 0$.

\renewcommand\qedsymbol{QED}
\end{proof}
\item $X=\{x\in\bbR\mid 0<x<1\}$
\begin{proof}
$\sup = 1$ and $\inf = 0$.

\renewcommand\qedsymbol{QED}
\end{proof}
\item  $X=\{3\}\cup \{x\in\bbR\mid -7\leq x\leq -5\}$ 
\begin{proof}
$\sup = 3$ and $\inf = -7$.

\renewcommand\qedsymbol{QED}
\end{proof}
\end{enumerate}
\end{exercise}



The following lemma is extremely useful when dealing with suprema; an analogous statement can be made for infima.
\begin{lemma} 
\label{lem1}
Suppose that $X$ is a subset of $C$ and $s = \sup X$ exists.
If $p<s$, then there exists an $x\in X$ such that $p < x \le s$.

\begin{proof}
Suppose there is no $x \in X$ such that $p<x$, then for all $x \in X$, $x \leq p$. Then, $p$ is an upper bound of $X$.

Since $p<s$, we have an upper bound less than $\sup X$, and this is a contradiction.

Hence, there must exist an $x \in X$ such that $p<x\leq s$.


\renewcommand\qedsymbol{QED}
\end{proof}
\end{lemma} 


\begin{theorem}  Let $a < b$.  The least upper bound and greatest lower bound of the region $\_{ab}$ are:
\[
\sup \_{ab} = b \quad \text{and} \quad \inf \_{ab} = a.
\]

\begin{proof}
For all $x \in \_{ab}, x \leq b$. Hence, $b$ is an upper bound of $\_{ab}$. Suppose $u \in \_{ab}$ is an upper bound such that $u < b$. Since regions are nonempty, there exists $m \in \_{ub}$, i.e., $u<m<b$, hence $u$ is not an upper bound. Thus, $\sup \_{ab} = b$.

For all $x \in \_{ab}, x \geq a$. Hence, $a$ is a lower bound of $\_{ab}$. Suppose $l \in \_{ab}$ is a lower bound such that $l >a$. Since regions are nonempty, there exists $m \in \_{al}$, i.e., $a<m<l$, hence $l$ is not a lower bound. Thus, $\inf \_{ab} = a$.

\renewcommand\qedsymbol{QED}
\end{proof}
\end{theorem}




\begin{lemma}  Let $C$ satisfy Axioms 1-3, but not necessarily Axiom 4. Suppose also that all regions of  $C$ are nonempty.
Let $X$ be a subset  of $ C$ and suppose  $\sup X$ exists. Then $\sup X\in \overline{X}.$ Similarly, $\inf X\in \overline{X}.$
\begin{proof}
We know that $\overline{X} = \{x \in C \mid \text{ for all regions } R \text{ containing } x, R \cap X \neq \emptyset\}$. Hence, it suffices to show that for all regions $R$ containing $\sup X$, $R \cap X \neq \emptyset$, and similarly for $\inf X$.

Let $R$ be a region containing $\sup X$. Then $R$ can be written as $\_{ab}$, such that $a<\sup X<b$. By Lemma 5.10, there exists $x \in X$ such that $a<x\leq\sup X$, hence $x \in R \cap X$, hence $R \cap X \neq \emptyset$.

\renewcommand\qedsymbol{QED}
\end{proof}
\end{lemma} 

\begin{corollary} Let $X$ be a subset of a connected continuum. Suppose that $\sup X$ exists. Then $\sup X\in \overline{X}.$ Similarly, $\inf X\in \overline{X}.$
\begin{proof}
By Corollary 5.2, all regions of a connected continuum are infinite, hence they are nonempty. Then, by Lemma 5.12, $\sup X \in \overline{X}$ and $\inf X \in \overline{X}$.

\renewcommand\qedsymbol{QED}
\end{proof}
\end{corollary}

\begin{corollary}  Both $a$ and $b$ are limit points of the region $\underline{ab}$.
\begin{proof}
We know that $a = \inf \_{ab}$ and $b = \sup \_{ab}$, hence $a, b \in \overline{\_{ab}}$.

Thus, $a, b \in \_{ab} \cup LP(\_{ab})$, but $a,b \notin \_{ab}$. Hence, $a, b \in LP(\_{ab})$.

\renewcommand\qedsymbol{QED}
\end{proof}
\end{corollary}

Let $[a, b]$ denote the closure $\overline{\_{ab}}$ of the region $\_{ab}$.  

\begin{corollary}  $[a, b] = \{x \in C \mid a \leq x \leq b  \}$.
\begin{proof}
We know that $a,b \in LP(\_{ab})$, hence $(\{a\}\cup\{b\}\cup\_{ab}) \subset [a, b]$.

Also, $C \setminus (\{a\}\cup\{b\}\cup\_{ab}) = \ext(\_{ab})$. We know that $LP(\_{ab}) \cap \ext(\_{ab}) = \emptyset$.

Hence, $[a, b] = \{a\}\cup\{b\}\cup\_{ab} = \{x \in C \mid a \leq x \leq b  \}$.

\renewcommand\qedsymbol{QED}
\end{proof}
\end{corollary}



\begin{lemma}  Let $X \subset C$ and define:
\[
\Psi(X) = \{ x \in C \mid \text{$x$ is not an upper bound of $X$} \}.
\]
Then $\Psi(X)$ is open.
Define:
\[
\Omega(X) = \{ x \in C \mid \text{$x$ is not a lower bound of $X$} \}.
\]
Then $\Omega(X)$ is open.
\begin{proof}
Let $x \in \Psi(X)$. Then, there exists some $y \in X$ such that $y>x$. 

Construct $\_{ay} = \{k \in C \mid a < k < y \}$ such that $a<x$ (since $C$ has no first point). Then, $x \in \_{ay}$.

It suffices to show that $\_{ay} 
\subset \Psi(X)$ (by Theorem 4.9). Let $k \in \_{ay}$, then $a<k<y$, hence there exists $y \in X$ such that $y>k$, hence $k$ is not an upper bound of $X$. Thus, $k \in \Psi(X)$.

Analogous reasoning for $\Omega(X)$ completes the proof.


\renewcommand\qedsymbol{QED}
\end{proof}
\end{lemma}


\begin{theorem}  Suppose that $X$ is nonempty and bounded above. Then $\sup X$ exists. Similarly, if $X$ is nonempty and bounded below, then $\inf X$ exists.
\begin{proof}
Let $A = \{a \in C \mid a \text{ is an upper bound of } X \}$. Since $X$ is bounded above, $A \neq \emptyset$ and we also know that $X \neq \emptyset$. 

Then, by the definition of upper bound, there exists $x\in X$ such that for all $a \in A$, $x \leq a$. Hence, $x$ is a lower bound of $A$.

Let $B = \{b \in C \mid b \text{ is a lower bound of } A \}$. Since $x \in B, B\neq \emptyset$. We know from Lemma 5.16 that $B$ is closed.

Let $k \in C$ and suppose $k \notin A$, i.e., $k$ is not an upper bound of $X$. Then, there exists $x_0 \in X$ such that $x_0>k$.

For all $a \in A$, $x_0 \leq a$, hence $k \leq a$, hence $k \in B$.

Let $s \in A \cap B$. We want to show that $s = \sup X$.

$s \in A$ implies $s$ is an upper bound of $X$. $s \in B$ implies $s$ is a lower bound of $A$. Hence, for all $a \in A$, $a \geq s$, where $a$ is an upper bound of $X$.

Hence, $s$ is the least upper bound of $X$, hence $s = \sup X$.

Analogous reasoning for $\inf X$ completes the proof.

\renewcommand\qedsymbol{QED}
\end{proof}
\end{theorem}

\begin{corollary}  Every nonempty closed and bounded set has a first point and a last point.
\begin{proof}
Let $X$ be nonempty, closed and bounded. Then, $X = \overline{X}$.

Since $X$ is bounded above and bounded below, $\sup X$ exists and $\inf X$ exists.

Since $\sup X, \inf X \in \overline{X}$, we have that $\sup X, \inf X \in X$. Hence, $\sup X$ is the last point of $X$ and $\inf X$ is the first point of $X$.

\renewcommand\qedsymbol{QED}
\end{proof}
\end{corollary}


\begin{exercise}  Is this true for $\mathbb{Q}$?
\begin{proof}

\renewcommand\qedsymbol{QED}
\end{proof}
\end{exercise}
\bigskip

\begin{center}
{\em Additional Exercises}
\end{center}

\begin{enumerate}
\item
For this exercise, we assume that $C=\bbR.$ Find $\sup X$ and $\inf X$ for each of the following subsets
of $\bbR,$ or state that they do not exist. You should give proofs.
\begin{enumerate}
\item $X=\{-2^n\mid n\in \bbN\}$
\item $X=\{ x\mid  a\leq x\leq b\}$
\item $X=\{-1\}\cup\{\frac{1}{n}\mid n\in\bbN\}$
\item $X=\bigcup_{k\geq 0}\{\frac{k}{n}\mid n\in\bbN\}.$
\end{enumerate}


\item Prove that if $A,B$ are subsets of a continuum $C$ and $a\leq b,$ for all $a\in A,b\in B,$ then
$$\sup A \leq \inf B.$$

\begin{proof}
Since for all $a\in A,b\in B, b\geq a$, we have that $b$ is an upper bound of $A$. By definition of sup, we know that $\sup A \leq b$ for all $b \in B$.

Since $b \geq \sup A$ for all $b \in B$, we know that $\sup A$ is a lower bound of $B$. Then, by definition of inf, we have that $\inf B \geq \sup A$. This completes the proof.


\renewcommand\qedsymbol{QED}
\end{proof}




\item
	Let $a,b \in C$ with $a < b$.  Show that $\underline{ab}$ is a continuum.
\begin{proof}
By Corollary 5.2, every region is infinite, hence $\_{ab}$ is nonempty. This satisfies Axiom 1.

Since $\_{ab} \subset C$, the ordering $<_C$ on $C$ restricts $\_{ab}$, i.e., $\_{ab}$ is equipped with the ordering $<_C$. This satisfies Axiom 2.

Suppose $\_{ab}$ has a first point, and let this be denoted by $m$. Then, we know that $a<m<b$. Consider the region $\_{am}$. Since every region is infinite, there exists some $x \in \_{am}$, i.e., $a<x<m$. This contradicts our original assumption that $\_{ab}$ has a first point. Analogous reasoning applies for the last point. This satisfies Axiom 3.

Suppose $\_{ab}$ is disconnected. Then we can write $\_{ab} = A\cup B$, where $A$ and $B$ are disjoint, nonempty and open.

For the first case, suppose there exist $u,v\in\underline{ab}$ such that $\underline{au}\subseteq A$ and $\underline{vb}\subseteq B$. Then we want to show that $A\cup\{x\in C \mid x\leq a\}$ and $B\cup\{x\in C \mid x\geq b\}$ are non-empty disjoint open subsets of $C$ whose union is $C$.

$a \in A\cup\{x\in C \mid x\leq a\}$ and $b \in B\cup\{x\in C \mid x\geq b\}$, hence they are nonempty.

Since $\underline{au} \subseteq A$, we have that $A\cup\{x\in C \mid x\leq a\} = A\cup\{x\in C \mid x<u\}$. This final expression is the union of two open sets, hence it is open. Analogously, $B\cup\{x\in C \mid x\geq b\}=B\cup\{x\in C \mid x>v\}$ is open.

Consider $y \in A$. By our original assumption, $A \cap B = \emptyset$, hence $y \notin B$. Also, since $\underline{vb} \subseteq B$, we have that $\underline{vb} \cap A = \emptyset$, hence $y \notin \{x\in C \mid x>v\}$. Next consider $y \in \{x\in C \mid x<u\}$. Since $A$ and $B$ are disjoint, we have that $u<v$, hence we know that $y \notin \{x\in C \mid x>v\}$. Since $\underline{au} \subseteq A$, hence $y \notin B$. Hence, we have that $A\cup\{x\in C \mid x<u\}$ and $B\cup\{x\in C \mid x>v\}$ are disjoint.

Take some $k \in C$. If $k \in \underline{ab}$, then $k \in A \cup B$. If $k \in C \setminus \underline{ab}$, then either $k \leq a$ in which case $k \in \{x\in C \mid x<u\}$, or $k \geq b$ in which case $k \in \{x\in C \mid x>v\}$. Hence, $A\cup\{x\in C \mid x<u\} \cup B\cup\{x\in C \mid x>v\} = C$. 

To consider the general case, let $u\in A$ and $v\in B$ and without loss of generality suppose that $u < v$. We define the sets $A' = \{x\in\underline{ab} \mid x\in A, x < v\}\cup\underline{au}$ and $B' = \{x\in\underline{ab} \mid x\in B, x > u\}\cup\underline{vb}$. We want to show that $A'$ and $B'$ are non-empty disjoint open subsets of $\underline{ab}$ whose union is $\underline{ab}$. This reduces the general case to the previous case.

We have that $A' = \{x\in\underline{ab} \mid x\in A, x < v\}\cup\underline{au} = \{A \cap \underline{av}\}\cup \underline{au}$. Analogously, $B' = \{x\in\underline{ab} \mid x\in B, x >u\}\cup\underline{vb} = \{B \cap \underline{ub}\}\cup \underline{vb}$. We know that $A, \underline{av}$ and $\underline{au}$ are open, hence $A'$ is open. Similarly, $B, \underline{ub}$ and $\underline{vb}$ are open, hence $B'$ is open.

$u \in A'$ and $v \in B'$, hence $A'$ and $B'$ are nonempty.

Consider $y \in A \cap \underline{av}$. $A$ and $B$ are disjoint, hence $y \notin B \cap \underline{ub}$, and $\underline{av} \cap \underline{vb} = \emptyset$, hence $y \notin B'$. Next consider $y \in \underline{au}$. We know that $\underline{au} \cap \underline{ub} = \emptyset$ and $\underline{au} \cap \underline{vb} = \emptyset$, hence $y \notin B'$. Thus, $A'$ and $B'$ are disjoint.

Hence, if $\underline{ab}$ is disconnected, then the continuum $C$ must be disconnected and this contradicts Axiom 4. Hence, $\underline{ab}$ is connected. This satisfies Axiom 4.

This completes the proof.

\renewcommand\qedsymbol{QED}
\end{proof}


\item
Assume that $C$ satisfies Axioms 1, 2 and 3. Do not assume that $C$ satisfies Axiom 4. Suppose also that
\begin{enumerate}
\item every nonempty subset $X$ of $C$ that is bounded above has a supremum 
\item all regions of $C$ are nonempty.
\end{enumerate}
Show that $C$ is connected. 

{\it Hint: Argue by contradiction. }

\begin{proof}
Suppose $C$ is disconnected. Then, we can write $C = A \cup B$, where $A,B$ are disjoint, nonempty and open.

Case 1: $A$ is bounded above. Then, by condition (a), we know that $\sup A$ exists. Let $s = \sup A$.

Subcase 1a: Suppose $s \in A$. Since $A$ is open, there exists a region $R \subset A$ such that $s \in R$. Let $R = \_{pq}$, then we have that $p<s<q$. Then consider the region $\_{sq} \subset A$. Since all regions are nonempty by condition (b), there exists $y \in \_{sq}$. Then, $y \in A, y>s$, and this contradicts the assumption that $s$ is the supremum of $A$.

Subcase 1b: Suppose $s \in B$. Since $B$ is open, there exists a region $R \subset B$ such that $s \in R$. Let $R = \_{pq}$, then we have that $p<s<q$. Then consider the region $\_{ps} \subset B$. Since all regions are nonempty by condition (b), there exists $y \in \_{ps}$, i.e. $p<y<s$. Then, we claim that $y$ is an upper bound of $A$, i.e., for all $a \in A, a \leq y$. Suppose there exists some $a_0$ such that $a_0 > y$. Since $s = \sup A, s \geq a_0$. But $a_0 \in A, s \in B$ hence $a_0 \neq s$, hence $s > a_0$. Then we have that $y<a_0<s$, i.e., $a \in \_{ys}$. However, $\_{ys} \subset B$ and this is a contradiction. 

This completes case 1. Next, we will show that the general case can be reduced to case 1.

Case 2: $A$ is not bounded above. Let $a \in A, b\in B$, and without loss of generality suppose $a <b$. Then, define new sets $A' = A \cap \{x \in C \mid x < b\}$ and $B' = B \cup \{x \in C \mid x > b\}$. We want to show that $A'$ is bounded above, and that $A'$ and $B'$ are nonempty, disjoint and open sets such that $A' \cup B' = C$. 

$B$ is open and $\{x \in C \mid x > b\}$ is open, hence $B'$ is the union of two open sets and is therefore open. $A$ is open and $\{x \in C \mid x < b\}$ is open, hence $A'$ is the intersection of two open sets and is therefore open. 

$b \in B$, hence $b \in B'$, hence $B'$ is nonempty. $a \in A$ and $a <b$ hence $a \in A'$, hence $A'$ is nonempty.

Suppose $m \in A'$. Then, $m \in A$ implies $m \notin B$ and $m \in \{x \in C \mid x < b\}$ implies $m \notin \{x \in C \mid x > b\}$. Hence, $m \notin B'$, thus $A' \cap B' = \emptyset$.

We can rewrite $A' = \{x \in A \mid x < b\}$. Hence, $b$ is an upper bound of $A'$, hence $A'$ is bounded above.

We can write $\{x \in C \mid x > b\} = \{x \in B \mid x > b\} \cup \{x \in A \mid x > b\}$. Then, $A' \cup B' = B \cup \{x \in A \mid x < b\} \cup \{x \in A \mid x > b\} = B \cup A = C$ (since $a \neq b$).

This completes the proof.

\renewcommand\qedsymbol{QED}
\end{proof}




\end{enumerate}






\end{document}